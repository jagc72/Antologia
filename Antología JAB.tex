\documentclass[11pt, portrait, twoside, notitlepage, openright]{book}
\usepackage[utf8]{inputenc}
\usepackage[spanish]{babel}
\usepackage{hyperref}
\usepackage{setspace}
\usepackage[left=3.0cm ,top=2.5cm, right=2.0cm, bottom=2.5cm]{geometry}

\title{\Huge{Antología de poesía de José Ángel Buesa.}}
\author{\Huge{}}
\date{febrero, 2019}
\begin{document}
\maketitle
\tableofcontents
\clearpage
\onehalfspacing


\thispagestyle{empty}

\newpage

\thispagestyle{empty}
\begin{verse}
\begin{center}
\section{Carta de amor.}
\end{center}
Aquí, sin ti,\\
ya sé lo que es la muerte,\\
pero no te lo digo\\
para no entristecerte.
\newline

Quiero que te sonrías\\
para que siga habiendo\\
claridad en los días.
\newline

Quiero que no\\
se empañe tu mirada,\\
pues, si no,\\
no habrá estrellas,\\
ni habrá luna, ni nada.
\newline

Y, sobre todo,\\
lo que quiero y quiero\\
es un año que tenga\\
doce meses de enero.

***\\

Aquí llueve y no importa,\\
pues la lluvia es tan leve\\
que al leer esta carta\\
no sentirás que llueve.
\newline

Pero cierro los ojos\\ 
y te recuerdo tanto\\
que casi se diría\\
que está lloviendo llanto.
\newline

\begin{flushright}
José Ángel Buesa.
\end{flushright}
\end{verse}

\doublespacing
\newpage
\begin{verse}
\begin{center}
\section{Carta de amor II.}
\end{center}

Y ya ves: yo estoy solo, murmurando tu nombre,\\
recordando los besos que te di y no te di,\\
y acaso tú, esta tarde, le sonreíste a un hombre\\
que ni siquiera se parece a mí.
\newline

O puede suceder, quién sabe cuándo,\\
que irás entre el gentío de una calle cualquiera,\\
y yo sé de qué modo se le quedan mirando\\
a una mujer bonita que pasa por la acera.
\newline

Sí, tal vez siento celos, celos tristes,\\
celos de no estar juntos, celos de no sé quién;\\
celos de por qué sales y de cómo te vistes,\\
que no quieren ser celos y son celos también.
\newline

Y de repente no te siento mía,\\
o estás como más lejos de repente,\\
y tengo la tristeza de una casa sombría\\
donde aún sopla el perfume de una mujer ausente.
\newpage

Afuera está la tarde, con su gris infinito;\\
afuera está la lluvia, calladamente cruel,\\
y quisiera decirte cómo te necesito...\\
pero se me emborrona la tinta en el papel...
\newline

\begin{flushright}
José Ángel Buesa.
\end{flushright}
\end{verse}

\newpage
\begin{verse}
\begin{center}
\section{Poema del domingo triste.}
\end{center}
Este domingo triste pienso en ti dulcemente\\
y mi vieja mentira de olvido, ya no miente.\\
La soledad, a veces, es peor castigo...\\
Pero, ¡qué alegre todo, si estuvieras conmigo!
\newline

Entonces no querría mirar las nubes grises,\\
formando extraños mapas de imposibles países;\\
y el monótono ruido del agua no sería\\
el motivo secreto de mi melancolía.
\newline

Este domingo triste nace de algo que es mío,\\
que quizás es tu ausencia y quizás es mi hastío,\\
mientras corren las aguas por la calle en declive\\
y el corazón se muere de un ensueño que vive.
\newline

La tarde pide un poco de sol, como un mendigo,\\
y acaso hubiera sol si estuvieras conmigo;\\
y tendría la tarde, fragantemente muda,\\
el ingenuo impudor de una niña desnuda.
\newline

Si estuvieras conmigo, amor que no volviste,\\
¡qué alegre me sería este domingo triste!
\newline

\begin{flushright}
José Ángel Buesa.
\end{flushright}
\end{verse}


\newpage
\begin{verse}
\begin{center}
\section{Balada del loco amor.}
\end{center}
\begin{center}
I   
\end{center}

No, nada llega tarde, porque todas las cosas\\
tienen su tiempo justo, como el trigo y las rosas;\\
sólo que, a diferencia de la espiga y la flor,\\
cualquier tiempo es el tiempo de que llegue el amor.
\newline

No, amor no llegas tarde. Tu corazón y el mío\\
saben secretamente que no hay amor tardío.\\
Amor, a cualquier hora, cuando toca a una puerta,\\
la toca desde adentro, porque ya estaba abierta.\\
Y hay un amor valiente y hay un amor cobarde,\\
pero, de cualquier modo, ninguno llega tarde.
\begin{center}
II
\end{center}

Amor, el niño loco de la loca sonrisa,\\
viene con pasos lentos igual que viene aprisa;\\
pero nadie está a salvo, nadie, si el niño loco\\
lanza al azar su flecha, por divertirse un poco.\\
Así ocurre que un niño travieso se divierte,\\
y un hombre, un hombre triste, queda herido de muerte.\\
Y más, cuando la flecha se le encona en la herida,\\
porque lleva el veneno de una ilusión prohibida.\\
Y el hombre arde en su llama de pasión, y arde, y arde,\\
y ni siquiera entonces el amor llega tarde.
\newpage
\begin{center}
III
\end{center}

No, yo no diré nunca qué noche de verano\\
me estremeció la fiebre de tu mano en mi mano.\\
No diré que esa noche que sólo a ti te digo\\
se me encendió en la sangre lo que soñé contigo.\\
No, no diré esas cosas, y, todavía menos,\\
la delicia culpable de contemplar tus senos.\\
Y no diré tampoco lo que vi en tu mirada,\\
que era como la llave de una puerta cerrada.\\
Nada más. No era el tiempo de la espiga y la flor,\\
y ni siquiera entonces llegó tarde el amor.
\newline
\begin{flushright}
José Ángel Buesa.
\end{flushright}
\end{verse}

\newpage
\begin{verse}
\begin{center}
\section{Canzonetta.}
\end{center}
Érase un verde bosque de eterna primavera,\\
y érase un niño iluso que vagaba al azar...\\
El niño entró en el bosque siguiendo una quimera;\\
entró en el bosque... y nadie lo ha visto regresar.
\newline

Érase un mar sereno, de tan hondo que era,\\
y érase un nauta loco que vio un día aquel mar...\\
El nauta aborrecía la paz de la ribera;\\
empuñó el remo... y nadie lo ha visto regresar.
\newline

Mujer: comprende el símil. Yo también quise un día\\
penetrar el secreto de tu melancolía,\\
y me perdí, y no pude regresar.
\newline

Porque en tus ojos verdes se extravió mi destino,\\
como el niño en el bosque, como el loco marino en el mar.
\newline

\begin{flushright}
José Ángel Buesa.
\end{flushright}
\end{verse}

\newpage
\begin{verse}
\begin{center}
\section{Último amor.}
\end{center}
Yo andaba entre la sombra, cuando como un fulgor\\
llegaste tú de pronto con el último amor.\\
Pero bastó un efluvio de antiguas primaveras\\
para reconocerte, para saber quien eras.
\newline

Y eras la misteriosa mujer desconocida,\\
que entristeció de ensueño lo mejor de mi vida.\\
La de las tardes grises y los claros de luna,\\
la que busqué entre tantas y no encontré en ninguna.
\newline

Y hoy tal vez como un premio, tal vez como un castigo,\\
lo mejor de mi vida será morir contigo.\\
He pensado esta noche, sintiéndote tan mía\\
que así como llegaste, pudieras irte un día.
\newline

Lo he pensado eso es todo. Pero si sucediera...\\
dejaré que te vayas sin un adiós siquiera.\\
Y cuando te hayas ido... yo que nunca me quejo,\\
me vestiré de luto y aprenderé a ser viejo.
\newpage

Pero si me muriera sin poder olvidarte\\
y después de la muerte se llega a alguna parte,\\
preguntaré si hay sitio para mí junto a ti,\\
y Dios seguramente responderá que sí.
\newline

José Ángel Buesa.
\end{verse}

\newpage
\begin{verse}
\begin{center}
\section{Canción de la lluvia.}
\end{center}
Acaso está lloviendo también en tu ventana;\\
Acaso esté lloviendo calladamente, así.\\
Y mientras anochece de pronto la mañana,\\
yo sé que, aunque no quieras, vas a pensar en mí.
\newline

Y tendrá un sobresalto tu corazón tranquilo,\\
sintiendo que despierta tu ternura de ayer.\\
Y, si estabas cosiendo, se hará un nudo en el hilo,\\
y aún lloverá en tus ojos, al dejar de llover.
\newline

José Ángel Buesa
\end{verse}

\newpage
\begin{verse}
\begin{center}
\section{Canción del amor prohibido.}
\end{center}
Sólo tú y yo sabemos lo que ignora la gente\\
al cambiar un saludo ceremonioso y frío,\\
porque nadie sospecha que es falso tu desvío,\\
ni cuánto amor esconde mi gesto indiferente.
\newline

Sólo tú y yo sabemos por qué mi boca miente,\\
relatando la historia de un fugaz amorío;\\
y tú apenas me escuchas y yo no te sonrío...\\
Y aún nos arde en los labios algún beso reciente.
\newline

Sólo tú y yo sabemos que existe una simiente\\
germinando en la sombra de este surco vacío,\\
porque su flor profunda no se ve, ni se siente.
\newline

Y así dos orillas tu corazón y el mío,\\
pues, aunque las separa la corriente de un río,\\
por debajo del río se unen secretamente.
\newline

José Ángel Buesa.
\end{verse}

\newpage
\begin{verse}
\begin{center}
\section{Canción del amor que pasa.}
\end{center}
Yo soy como un viajero que no duerme\\
más de una vez en una misma casa.\\
Dame un beso y olvídame. No intentes retenerme:\\
Soy el amor que pasa.
\newline

Yo soy como una nube que da sombra un instante;\\
soy una hoguera efímera que no deja brasa.\\
Yo soy el buen amor y el mal amante.\\
Dime adiós y sonríeme. Soy el amor que pasa.
\newline

Soy el amor que olvida, pero que nunca miente;\\
que muere sonriendo porque nace feliz;\\
yo paso como un ala, fugazmente;\\
y, aunque se siembre un ala, nunca tendrá raíz.\\
No intentes retenerme déjame que me vaya\\
como el agua de un río, que no vuelve a pasar...
\newpage

Yo soy como una ola en la playa,\\
pues las olas se acercan, pero vuelven al mar...\\
Soy el amor de amar, que odia lo inerme,\\
que se lleva el perfume, pero deja la flor...
\newline

Dime adiós y no intentes retenerme.\\
Soy el amor que pasa... ¡pero soy el amor!
\newline

José Ángel Buesa.
\end{verse}

\newpage
\begin{verse}
\begin{center}
\section{Canción de la noche sola.}
\end{center}
\begin{center}
I
\end{center}

Fue mía una noche. Llegó de repente,\\
y huyó como el viento, repentinamente.

Alumna curiosa que aprendió el placer,\\
fue mía una noche. No la he vuelto a ver.

Fue la noche sola de una sola estrella.\\
Si miro las nubes, después pienso en ella.

Mi amor no la busca; mi amor no la llama:\\
La flor desprendida no vuelve a la rama,

y las ilusiones son como un espejo\\
que cuando se empaña pierde su reflejo.

\begin{center}
II
\end{center}

Fue mía una noche, locamente mía:\\
Me quema los labios su sed todavía.

Bella como pocas, nunca fue más bella\\
que soñando el sueño de la noche aquella.

Su amor de una noche sigue siendo mío:\\
La corriente pasa, pero queda el río;

y si ella es la estrella de una noche sola,\\
yo he sido en su playa la primera ola.
\newpage
\begin{center}
III
\end{center}

Amor de una noche que ignoró el hastío:\\
Somos las distantes orillas de un río,

entre las que cruza la corriente clara,\\
y el agua las une, pero las separa.

Amor de una noche: si vuelves un día,\\
ya no he de sentirte tan loca y tan mía.

Más que la tortura de una herida abierta,\\
mi amor ama el viento que cierra una puerta.

El amor florece tierra movediza,\\
y es ley de la llama trocarse en ceniza.

El amor que vuelve siempre vuelve en vano,\\
así como un ciego que extiende la mano.

Amor de una noche sin amanecer:\\
¡Acaso prefiero no volverte a ver!

José Ángel Buesa.
\end{verse}

\newpage
\begin{verse}
\begin{center}
\section{Canción de la espera.}
\end{center}
Espero tu sonrisa y espero tu fragancia\\
por encima de todo, del tiempo y la distancia.\\
Yo no sé desde dónde, hacia dónde, ni cuándo\\
regresarás... sé sólo que te estaré esperando.
\newline

En lo alto del bosque y en lo hondo del lago,\\
en el minuto alegre y en el minuto aciago,\\
en la función pagana y en el sagrado rito,\\
en el limpio silencio y en el áspero grito.
\newline

Allí donde es más fuerte la voz de la cascada,\\
allí donde está todo y allí donde no hay nada,\\
en la pluma del ala y en el sol del ocaso,\\
yo esperaré el sonido rítmico de tu paso.
\newline

Comprendo que de mí ya se ría la gente\\
al ver cómo te espero desesperadamente.\\
Cuando todos los astros se apaguen en el cielo,\\
cuando todos los pájaros paralicen el vuelo\\
cansados de esperarte, ese día\\
lejano yo te estaré esperando todavía.
\newpage

No importa: aunque me digan todos que desvarío,\\
yo te espero en las ondas musicales del río,\\
en la nube que llega blanca de su trayecto,\\
en el camino angosto y en el camino recto.
\newline

Niño, joven o anciano, sonriendo o llorando,\\
en el alba o la tarde, yo te estaré esperando,\\
y si me convenciera que ese ansiado día\\
no habría de llegar, también te esperaría.
\newline

José Ángel Buesa.
\end{verse}

\newpage
\begin{verse}
\begin{center}
\section{Canción del amor lejano.}
\end{center}
Ella no fue, entre todas, la más bella,\\
pero me dio el amor más hondo y largo.\\
Otras me amaron más; y, sin embargo,\\
a ninguna la quise como a ella.
\newline

Acaso fue porque la amé de lejos,\\
como una estrella desde mi ventana...\\
Y la estrella que brilla más lejana\\
nos parece que tiene más reflejos.
\newline

Tuve su amor como una cosa ajena\\
como una playa cada vez más sola,\\
que únicamente guarda de la ola\\
una humedad de sal sobre la arena.
\newline

Ella estuvo en mis brazos sin ser mía,\\
como el agua en cántaro sediento,\\
como un perfume que se fue en el viento\\
y que vuelve en el viento todavía.
\newline

Me penetró su sed insatisfecha\\
como un arado sobre llanura,\\
abriendo en su fugaz desgarradura\\
la esperanza feliz de la cosecha.
\newpage
Ella fue lo cercano en lo remoto,\\
pero llenaba todo lo vacío,\\
como el viento en las velas del navío,\\
como la luz en el espejo roto.
\newline

Por eso aún pienso en la mujer aquella,\\
la que me dio el amor más hondo y largo...\\
Nunca fue mía. No era la más bella.\\
Otras me amaron más... Y, sin embargo,\\
a ninguna la quise como a ella.
\newline

José Ángel Buesa.
\end{verse}

\newpage
\begin{verse}
\begin{center}
\section{Balada del mal amor.}
\end{center}
Qué lástima, muchacha,\\
que no te pueda amar...\\
Yo soy un árbol seco que sólo espera el hacha,\\
y tú un arroyo alegre que sueña con la mar.
\newline

Yo eché mi red al río...\\
Se me rompió la red...\\
No unas tu vaso lleno con mi vaso vacío,\\
pues si bebo en tu vaso voy a sentir más sed.
\newline

Se besa por el beso,\\
por amar el amor...\\
Ese es tu amor de ahora, pero el amor no es eso;\\
pues sólo nace el fruto cuando muere la flor.
\newline

Amar es tan sencillo,\\
tan sin saber por qué...\\
Pero así como pierde la moneda su brillo,\\
el alma, poco a poco, va perdiendo su fe.
\newline

¡Qué lástima muchacha,\\
que no te pueda amar!\\
Hay velas que se rompen a la primera racha,\\
¡y hay tantas velas rotas en el fondo del mar!
\newline

Pero aunque toda herida\\
deja una cicatriz,\\
no importa la hoja seca de una rama florida,\\
si el dolor de esa hoja no llega a la raíz.
\newline

La vida, llama o nieve,\\
es un molino que\\
va moliendo en sus aspas el viento que lo mueve,\\
triturando el recuerdo de lo que ya se fue...
\newline

Ya lo mío fue mío,\\
y ahora voy al azar...\\
Si una rosa es más bella mojada de rocío,\\
el golpe de la lluvia la puede deshojar...
\newline

Tuve un amor cobarde.\\
Lo tuve y lo perdí...\\
Para tu amor temprano ya es demasiado tarde,\\
porque en mi alma anochece lo que amanece en ti.
\newline

El viento hincha la vela, pero la deshilacha,\\
y el agua de los ríos se hace amarga en el mar...\\
Qué lástima muchacha,\\
que no te pueda amar...
\newline
\begin{flushright}
José Ángel Buesa.
\end{flushright}
\end{verse}

\newpage
\begin{verse}
\begin{center}
\section{Brindis.}
\end{center}
He aquí dos rosas frescas, mojadas de rocío:\\
una blanca, otra roja, como tu amor y el mío.\\
Y he aquí que, lentamente, las dos rosas deshojo:\\
la roja, en vino blanco; la blanca, en vino rojo.
\newline

Al beber, gota a gota, los pétalos flotantes\\
me rozarán los labios, como labios de amante;\\
y, en su llama o su nieve de idéntico destino,\\
serán como fantasmas de besos en el vino.
\newline

Ahora, elige tú, amiga, cuál ha de ser tu vaso:\\
si éste, que es como un alba, o aquél, como un ocaso.\\
No me preguntes nada: yo sé bien que es mejor
\newline

embriagarse de vino que embriagarse de amor...\\
Y así mientras tú bebes, sonriéndome así,\\
yo, sin que tú lo sepas, me embriagaré de ti...
\newline

\begin{flushright}
José Ángel Buesa.
\end{flushright}
\end{verse}
\newpage
\begin{verse}
\begin{center}
\section{Canción a la mujer lejana.}
\end{center}
En ti recuerdo una mujer lejana,\\
lejana de mi amor y de mi vida.\\
A la vez diferente y parecida,\\
como el atardecer y la mañana.
\newline

En ti despierta esa mujer que duerme\\
con tantas semejanzas misteriosas,\\
que muchas veces te pregunto cosas,\\
que sólo ella podría responderme.
\newline

Y te digo que es bella, porque es bella,\\
pero no sé decir, cuando lo digo,\\
si pienso en ella porque estoy contigo,\\
o estoy contigo por pensar en ella.
\newline

Y sin embargo si el azar mañana\\
me enfrenta con ella de repente,\\
no seguiría a la mujer ausente\\
por retener a la mujer cercana.
\newpage

Y sin amarte más, pero tampoco\\
sin separar tu mano de la mía,\\
al verla simplemente te diría:\\
``Esa mujer se te parece un poco"
\newline

\begin{flushright}
José Ángel Buesa.
\end{flushright}
\end{verse}

\newpage
\begin{verse}
\begin{center}
\section{Canción de un sueño.}
\end{center}

Otra vez, esta noche, vi tu mano en la mía,\\
otra vez, esta noche, volví a soñar contigo, \\
yo, que no soy tu amante ni siquiera tu amigo, \\
sino un hombre que pasa bajo la luz del día. 
\newline

Sin embargo, en la sombra donde el tiempo no existe, \\
se buscan nuestras almas, no sé por qué. Y despierto\\
vagamente inconforme de que no ha sido cierto, \\
triste de una tristeza que no llega a ser triste.
\newline

Algo ocurre en la noche, pero yo no lo digo: \\
ni a ti, que nada sabes, ni a ti te diré nada, \\
pero al mirar tus ojos sabré, por tu mirada,\\
si también, esta noche, tú has soñado conmigo.
\newline

\begin{flushright}
José Ángel Buesa.
\end{flushright}
\end{verse}

\newpage
\begin{verse}
\begin{center}
\section{Ella amará a otro hombre.}
\end{center}
Ella amará a otro hombre.\\
Yo voy lejos, andando hacia el olvido.\\
Y puede suceder que alguien me nombre\\
pero ella fingirá no haber oído.
\newline

Ella amará a otro hombre: el tiempo pasa\\
y el amor finaliza,\\
y es natural que lo que fue una brasa\\
acabe convirtiéndose en ceniza.
\newline

Aunque nadie lo quiera,\\
envejecen las vidas y las cosas,\\
y es natural también que en primavera\\
los rosales den rosas.
\newline

Es natural. Por eso,\\
ella amará a otro hombre, y está bien.\\
No sé si ya olvidó mi último beso,\\
ni me importa con quién.
\newline

Pero quizás, un día,\\
oyendo una canción,\\
sentirá que esa vieja melodía\\
le cambia el ritmo de su corazón.
\newpage

O será algún vestido\\
que yo le conocí,\\
o el olor del jardín cuando ha llovido,\\
pero algún día ha de pensar en mí.
\newline

O puede ser un gesto,\\
un modo de mirar,\\
o ciertas calles, o un botón mal puesto,\\
o una hoja seca que voló al azar.
\newline

Y de alguna manera\\
tendrá que recordarme, sin querer,\\
escuchando unos pasos en la acera\\
como los míos al atardecer.
\newline

Será en algún momento,\\
no importa cuándo o dónde, aquí o allá,\\
porque el amor, por parecerse al viento,\\
parece que se ha ido y no se va.
\newline

Y si en ese momento ella suspira\\
y él pregunta por qué,\\
le tendrá que inventar una mentira\\
para que nunca sepa por qué fue.
\newpage

Y él no verá esa huella,\\
eso tan mío en lo que ya perdí;\\
Y, aunque la pueda amar más que yo a ella,\\
ella no podrá amarlo más que a mí...
\newline

\begin{flushright}
José Ángel Buesa.
\end{flushright}
\end{verse}

\newpage
\begin{verse}
\begin{center}
\section{Soneto lloviendo.}
\end{center}

No hace falta que llueva como llueve este día,\\
y, sin embargo, llueve desde el amanecer.\\
Si hay rosas y retoños, ¿para qué llovería?\\
Si ya todo florece, ¿qué más va a florecer?
\newline

Llueve obstinadamente y en la calle vacía\\
las gotas de la lluvia son pasos de mujer.\\
Pero cierro los ojos y llueve todavía\\
y al abrirlos de nuevo no deja de llover.
\newline

Yo sé que no hace falta que llueva, pero llueve.\\
Y recuerdo una tarde maravillosa y breve,\\
que fue maravillosa porque llovía así...
\newline

Y es tan triste, tan triste, la lluvia en mi ventana,\\
que casi me pregunto, dulce amiga lejana,\\
si no estará lloviendo para que piense en ti.
\newline

\begin{flushright}
José Ángel Buesa.
\end{flushright}
\end{verse}

\newpage
\begin{verse}
\begin{center}
\section{Mejor no quiero verte.}
\end{center}
Mejor no quiero verte... Sería tan sencillo\\
cruzar dos o tres calle... Y tocar en tu puerta.\\
Y tú me mirarías con tus ojos sin brillo\\
sin poder sonreírme con tu sonrisa muerta.
\newline

Mejor no quiero verte... porque va a hacerme daño\\
pasar por aquel parque de la primera cita.\\
Y no sé si aún florecen los jazmines de antaño\\
ni sé quién es ahora la mujer más bonita.
\newline

Mejor no quiero verte... porque andando en tu acera\\
sentiré casi ajeno todo lo que fue mío.\\
Aunque es sólo una esquina donde nadie me espera\\
y unos cristales rotos en un balcón vacío.
\newline

Sí... seguiré muriendo de mi pequeña muerte\\
de hace ya tantos años el día que me fui\\
pues por no verte vieja... mejor no quiero verte,\\
pero tampoco quiero que me veas tú a mí.
\newline

\begin{flushright}
José Ángel Buesa.
\end{flushright}
\end{verse}

\newpage
\begin{verse}
\begin{center}
\section{Poema vulgar.}
\end{center}
La vi pasar con otro... su semblante\\
resplandecía de felicidad.\\
Y me subió a los labios mi sonrisa galante,\\
con algo de impotencia y algo de vanidad.
\newline

En las manos del otro palpitaban sus manos;\\
en el brazo del otro se apoyaba feliz...\\
Y me envolvió una niebla de recuerdos lejanos,\\
y sentí que sangraba mi vieja cicatriz.
\newline

La vi pasar con otro, risueña y arrogante.\\
Me pareció más bella, más gallarda... No sé.\\
Sólo sé que de nuevo la amé en aquel instante,\\
más que cuando fue mía, si es que entonces la amé...
\newline

Y, de esa llamarada que aún me quema,\\
de ese dolor amargo como un golpe de mar,\\
ya lo veis: ha nacido este poema\\
deplorablemente vulgar...
\newline

José Ángel Buesa.
\end{verse}

\newpage
\begin{verse}
\begin{center}
\section{Poema de la culpa.}
\end{center}
Yo la amé, y era de otro, que también la quería.\\
Perdónala Señor, porque la culpa es mía.
\newline

Después de haber besado sus cabellos de trigo,\\
nada importa la culpa, pues no importa el castigo.
\newline

Fue un pecado quererla, Señor, y, sin embargo\\
mis labios están dulces por ese amor amargo.
\newline

Ella fue como un agua callada que corría...\\
Si es culpa tener sed, toda la culpa es mía.\\
Perdónala Señor, tú que le diste a ella\\
su frescura de lluvia y esplendor de estrella.
\newline

Su alma era transparente como un vaso vacío.\\
Yo lo llené de amor. Todo el pecado es mío.
\newline

Pero, ¿cómo no amarla, si tú hiciste que fuera\\
turbadora y fragante como la primavera?
\newline

¿Cómo no haberla amado, si era como el rocío\\
sobre la yerba seca y ávida del estío?

Traté de rechazarla, Señor, inútilmente,\\
como un surco que intenta rechazar la simiente.
\newline

Era de otro. Era de otro, que no la merecía,\\
y por eso, en sus brazos, seguía siendo mía.
\newline

Era de otro, Señor. Pero hay cosas sin dueño:\\
Las rosas y los ríos, y el amor y el ensueño.
\newline

Y ella me dio su amor como se da una rosa,\\
como quien lo da todo, dando tan poca cosa...
\newline

Una embriaguez extraña nos venció poco a poco:\\
ella no fue culpable, Señor... ¡ni yo tampoco!
\newline

La culpa es toda tuya, porque la hiciste bella\\
y me diste los ojos para mirarla a ella.
\newline

Toda la culpa es tuya, pues me hiciste cobarde\\
para matar un sueño porque llegaba tarde.
\newline

Sí. Nuestra culpa es tuya, si es una culpa amar\\
y si es culpable un río cuando corre hacia el mar.
\newline

Es tan bella, Señor, y es tan suave, y tan clara,\\
que sería un pecado mayor si no la amara.
\newpage

Y, por eso, perdóname, Señor, porque es tan bella,\\
que tú que hiciste el agua, y la flor, y la estrella,
\newline

tú, que oyes el lamento de este dolor sin nombre,\\
¡tú también la amarías, si pudieras ser hombre!
\newline

José Ángel Buesa.
\end{verse}

\newpage
\begin{verse}
\begin{center}
\section{Poema del renunciamiento.}
\end{center}
Pasarás por mi vida sin saber que pasaste.\\
Pasarás en silencio por mi amor, y al pasar,\\
fingiré una sonrisa, como un dulce contraste\\
del dolor de quererte... y jamás lo sabrás.
\newline

Soñaré con el nácar virginal de tu frente;\\
soñaré con tus ojos de esmeraldas de mar;\\
soñaré con tus labios desesperadamente;\\
soñaré con tus besos... y jamás lo sabrás.
\newline

Quizás pases con otro que te diga al oído\\
esas frases que nadie como yo te dirá;\\
y, ahogando para siempre mi amor inadvertido,\\
te amaré más que nunca... y jamás lo sabrás.
\newline

Yo te amare en silencio, como algo inaccesible,\\
como un sueño que nunca lograré realizar;\\
y el lejano perfume de mi amor imposible\\
rozará tus cabellos... y jamás lo sabrás.
\newpage

Y si un día una lágrima denuncia mi tormento,\\
-- el tormento infinito que te debo ocultar --\\
te diré sonriente: ``No es nada ... ha sido el viento".\\
Me enjugaré la lágrima... ¡y jamás lo sabrás!
\newline

José Ángel Buesa.
\end{verse}

\newpage
\begin{verse}
\begin{center}
\section{Poema de la despedida.}
\end{center}
Te digo adiós si acaso te quiero todavía\\
Quizás no he de olvidarte... Pero te digo adiós\\
No se si me quisiste... No se si te quería\\
O tal vez nos quisimos demasiado los dos.
\newline

Este cariño triste y apasionado y loco\\
Me lo sembré en el alma para quererte a ti.\\
No se si te amé mucho... No se si te amé poco,\\
Pero si sé que nunca volveré a amar así.
\newline

Me queda tu sonrisa dormida en mi recuerdo\\
Y el corazón me dice que no te olvidaré.\\
Pero al quedarme solo... Sabiendo que te pierdo,\\
tal vez empiezo a amarte como jamás te amé.
\newline

Te digo adiós y acaso con esta despedida\\
Mi más hermoso sueño muere dentro de mí.\\
Pero te digo adiós para toda la vida,\\
Aunque toda la vida siga pensando en ti.
\newline

José Ángel Buesa.
\end{verse}

\newpage
\begin{verse}
\begin{center}
\section{Carta sin fecha.}
\end{center}
Amigo: sé que existes, pero ignoro tu nombre.\\
No lo he sabido nunca ni lo quiero saber.\\
Pero te llamo amigo para hablar de hombre a hombre,\\
que es el único modo de hablar de una mujer.
\newline

Esa mujer es tuya, pero también es mía.\\
Si es más mía que tuya, lo saben ella y Dios.\\
Sólo sé que hoy me quiere como ayer te quería,\\
aunque quizá mañana nos olvide a los dos.
\newline

Ya ves: ahora es de noche. yo te llamo mi amigo;\\
yo, que aprendí a estar solo para quererla más;\\
y ella, en tu propia almohada, tal vez sueña conmigo;\\
y tú, que no lo sabes, no la despertarás.
\newline

¡Qué importa lo que sueña!  Déjala así, dormida.\\
Yo seré como un sueño sin mañana ni ayer.\\
Y ella irá de tu brazo para toda la vida,\\
y abrirá las ventanas en el atardecer.
\newline

Quédate tú con ella. Yo seguiré el camino.\\
Ya es tarde, tengo prisa, y aún hay mucho que andar,\\
y nunca rompo el vaso donde bebí un buen vino,\\
ni siembro nada, nunca, cuando voy hacia el mar.
\newpage

Y pasarán los años favorables o adversos,\\
y nacerán las rosas que nacen porque sí;\\
y acaso tú, algún día, leerás estos versos,\\
sin saber que los hice por ella y para ti...
\newline

José Ángel Buesa.
\end{verse}

\newpage
\begin{verse}
\begin{center}
\section{Poema del secreto.}
\end{center}
Puedo tocar tu mano sin que tiemble la mía,\\
y no volver el rostro para verte pasar.\\
Puedo apretar mis labios un día y otro día.\\
y no puedo olvidar. 
\newline

Puedo mirar tus ojos y hablar frívolamente,\\
casi aburridamente, sobre un tema vulgar,\\
puedo decir tu nombre con voz indiferente...\\
y no puedo olvidar.
\newline

Puedo estar a tu lado como si no estuviera,\\
y encontrarte cien veces, así como al azar...\\
puedo verte con otro, sin suspirar siquiera,\\
y no puedo olvidar.
\newline

Ya ves: Tu no sospechas este secreto amargo,\\
más amargo y profundo que el secreto del mar...\\
porque puedo dejarte de amar, y sin embargo...\\
¡no te puedo olvidar!
\newline

José Ángel Buesa.
\end{verse}

\newpage
\begin{verse}
\begin{center}
\section{Poema del amor imposible.}
\end{center}

Esta noche pasaste por mi camino\\
y me tembló en el alma no sé qué afán\\
pero yo estoy consciente de mi destino\\
que es mirarte de lejos y nada más.
\newline

No, tú nunca dijiste que hay primavera\\
en las rosas ocultas de tu rosal.\\
Ni yo debo mirarte de otra manera\\
que mirarte de lejos y nada más.
\newline

Y así pasas a veces tranquila y bella,\\
así como esta noche te vi pasar.\\
Más yo debo mirarte como una estrella\\
que se mira de lejos y nada más.
\newline

Y así pasan las rosas de cada día\\
dejando las raíces que no se van.\\
Y yo con mi secreta melancolía\\
de mirarte de lejos y nada más.
\newpage

Y así seguirás siempre, siempre prohibida,\\
más allá de la muerte, si hay más allá.\\
Porque en esa vida, si hay otra vida,\\
te miraré de lejos y nada más...
\newline

José Ángel Buesa.
\end{verse}

\newpage
\begin{verse}
\begin{center}
\section{Acuérdate de mi.}
\end{center}
Cuando vengan las sombras del olvido\\
a borrar de mi alma el sentimiento,\\
no dejes, por Dios, borrar el nido\\
donde siempre durmió mi pensamiento.
\newline

Si sabes que mi amor jamás olvida\\
que no puedo vivir lejos de ti\\
dime que en el sendero de la vida\\
alguna vez te acordarás de mí.
\newline

Cuando al pasar inclines la cabeza\\
y yo no pueda recoger tu llanto,\\
en esa soledad de la tristeza\\
te acordarás de aquel que te amó tanto.
\newline

No podrás olvidar que te he adorado\\
con ciego y delirante frenesí\\
y en las confusas sombras del pasado,\\
luz de mis ojos, te acordarás de mí.
\newpage

El tiempo corre con denso vuelo\\
ya se va adelantando entre los dos\\
no me olvides jamás. ¡Dame un recuerdo!\\
y no me digas para siempre adiós.
\newline

José Ángel Buesa.
\end{verse}


\newpage
\begin{verse}
\begin{center}
\section{Elegía lamentable.}
\end{center}
Desde este mismo instante seremos dos extraños\\
por estos pocos días, quien sabe cuantos años...\\
yo seré en tu recuerdo como un libro prohibido\\
uno de esos que nadie confiesa haber leído.\\
Y así mañana, al vernos en la calle, al ocaso,\\
tú bajarás los ojos y apretarás el paso,\\
y yo, discretamente, me cambiaré de acera,\\
o encenderé un cigarro, como si no te viera...
\newline

Seremos dos extraños desde este mismo instante\\
y pasarán los meses, y tendrás otro amante:\\
y como eres bonita, sentimental y fiel,\\
quizás, andando el tiempo, te casaras con el.\\
Y ya, más que un esposo será como un amigo,\\
aunque nunca le cuentes que has soñado conmigo,\\
y aunque, tras tu sonrisa, de mujer satisfecha,\\
se te empañen los ojos, al llegar una fecha.
\newpage

Acaso, cuando llueva, recordarás un día\\
en que estuvimos juntos y en que también llovía.\\
Y quizás nunca más te pongas aquel traje\\
de terciopelo verde, con adornos de encaje.\\
O harás un gesto mío, tal vez sin darte cuenta,\\
cuando dobles tu almohada con mano soñolienta.\\
Y domingo a domingo, cuando vayas a Misa,\\
de tu casa a la Iglesia, perderás tu sonrisa.
\newline

¿Qué más puedo decirte? Serás la esposa honesta\\
que abanica al marido cuando ronca la siesta:\\
tras fregar los platos y tender las camas,\\
te pasarás las noches sacando crucigramas...\\
y así, años y años, hasta que, finalmente\\
te morirás un día, como toda la gente.\\
Y voces que aún no existen sollozarán tu nombre,\\
y cerrarán tus ojos los hijos de otro hombre.
\newline

José Ángel Buesa
\end{verse}

\newpage
\begin{verse}
\begin{center}
\section{Elegía para ti y para mi.}
\end{center}
Yo seguiré soñando mientras pasa la vida,\\
y tú te irás borrando lentamente de mi sueño.\\
Un año y otro año caerán como hojas secas\\
de las ramas del árbol milenario del tiempo,\\
y tu sonrisa, llena de claridad de aurora,\\
se alejará en la sombra creciente del recuerdo.
\newline

Yo seguiré soñando mientras pasa la vida,\\
y quizá, poco a poco, dejaré de hacer versos,\\
bajo el vulgar agobio de la rutina diaria,\\
de las desilusiones y los aburrimientos.\\
Tú, que nunca soñaste mas que cosas posibles,\\
dejarás, poco a poco, de mirarte al espejo.
\newline

Acaso nos veremos un día, casualmente,\\
al cruzar una calle, y nos saludaremos.\\
Yo pensaré quizá: ``Qué linda es todavía."\\
Tú quizá pensarás: ``Se está poniendo viejo"\\
Tú irás sola, o con otro. Yo iré solo o con otra.\\
o tú irás con un hijo que debiera ser nuestro.
\newpage

Y seguirá muriendo la vida, año tras año,\\
igual que un río oscuro que corre hacia el silencio.\\
Un amigo, algún día, me dirá que te ha visto,\\
o una canción de entonces me traerá tu recuerdo.\\
Y en estas noches tristes de quietud y de estrellas,\\
pensaré en ti un instante, pero cada vez menos...
\newline

Y pasará la vida. Yo seguiré soñando;\\
pero ya no habrá un nombre de mujer en mi sueño.\\
Yo ya te habré olvidado definitivamente\\
y sobre mis rodillas retozarán mis nietos.\\
(Y quizá, para entonces, al cruzar una calle,\\
nos vimos frente a frente, ya sin reconocernos.)
\newline

José Ángel Buesa.
\end{verse}

\newpage
\begin{verse}
\begin{center}
\section{Amor insatisfecho.}
\end{center}
Mi corazón se siente satisfecho\\
de haberte amado y nunca poseído;\\
así tu amor se salva del olvido\\
igual que mi ternura del despecho.
\newline

Jamás te vi desnuda sobre el lecho,\\
ni oí tu voz muriéndose en mi oído;\\
así ese bien fugaz no ha convertido\\
un ancho amor en un placer estrecho.
\newline

Cuanto el deleite suma a lo vivido\\
acrecentado se lo resta el pecho,\\
pues la ilusión se va por el sentido.
\newline

Y en ese hacer y deshacer lo hecho,\\
sólo un amor se salva del olvido,\\
y es el amor que queda insatisfecho.
\newline

José Ángel Buesa
\end{verse}

\newpage
\begin{verse}
\begin{center}
\section{Se deja de querer.}
\end{center}
Se deja de querer...\\
y no se sabe por qué se deja de querer;\\
es como abrir la mano y encontrarla vacía\\
y no saber de pronto qué cosa se nos fue.
\newline

Se deja de querer...\\
y es como un río cuya corriente fresca ya no calma la sed,\\
como andar en otoño sobre las hojas secas\\
y  pisar la hoja verde que no debió caer.
\newline

Se deja de querer...\\
Y es como el ciego que aún dice adiós llorando\\
después que pasó el tren,\\
o como quien despierta recordando un camino\\
pero ya sólo sabe que regresó por él.
\newline

Se deja de querer...\\
como quien deja de andar una calle sin razón, sin saber,\\
y es hallar un diamante brillando en el rocío\\
y que ya al recogerlo se evapore también.
\newpage

Se deja de querer...\\
y es como un viaje detenido en las sombras\\
sin seguir ni volver,\\
y es cortar una rosa para adornar la mesa\\
y que el viento deshoje la rosa en el mantel.
\newline

Se deja de querer...\\
y es como un niño que ve cómo naufragan sus barcos de papel,\\
o escribir en la arena la fecha de mañana\\
y que el mar se la lleve con el nombre de ayer.
\newline

Se deja de querer...\\
y es como un libro que aún abierto hoja a hoja quedó a medio leer,\\
y es como la sortija que se quitó del dedo\\
y solo así supimos... que se marcó en la piel.
\newline

Se deja de querer...\\
y no se sabe por qué se deja de querer.
\newline

José Ángel Buesa.
\end{verse}

\newpage
\begin{verse}
\begin{center}
\section{La dama de las perlas.}
\end{center}
Yo he visto perlas claras de inimitable encanto,\\
de esas que no se tocan por temor a romperlas.\\
Pero sólo en tu cuello pudieron valer tanto\\
las burbujas de nieve de tu collar de perlas.
\newline

Y más, aquella noche del amor satisfecho,\\
del amor que eterniza lo fugaz de las cosas,\\
cuando fuiste un camino que comenzó en mi lecho\\
y el rubor te cubría como un manto de rosas.
\newline

Yo acaricié tus perlas, sin desprender su broche,\\
y las vi, como nadie nunca más podrá verlas,\\
pues te tuve en mis brazos, al fin, aquella noche\\
vestida solamente ¡con tu collar de perlas!
\newline

José Ángel Buesa
\end{verse}

\newpage
\begin{verse}
\begin{center}
\section{Poema de las cosas.}
\end{center}

Quizás estando sola, de noche, en tu aposento\\
oirás que alguien te llama sin que tu sepas quién\\
y aprenderás entonces, que hay cosas como el viento\\
que existen ciertamente, pero que no se ven...
\newline

Y también es posible que una tarde de hastío\\
como florece un surco, te renazca un afán\\
y aprenderás entonces que hay cosas como el río\\
que se están yendo siempre, pero que no se van...
\newline

O al cruzar una calle, tu corazón risueño\\
recordará una pena que no tuviste ayer\\
y aprenderás entonces que hay cosas como el sueño,\\
cosas que nunca han sido, pero que pueden ser...
\newline

Por más que tú prefieras ignorar estas cosas\\
sabrás por qué suspiras oyendo una canción\\
y aprenderás entonces que hay cosas como rosas,\\
cosas que son hermosas, sin saber que lo son...
\newpage

Y una tarde cualquiera, sentirás que te has ido\\
y un soplo de ceniza regará tu jardín\\
y aprenderás entonces, que el tiempo y el olvido\\
son las únicas cosas que nunca tienen fin.
\newline

José Ángel Buesa.
\end{verse}

\newpage
\begin{verse}
\begin{center}
\section{Así, verte de lejos.}
\end{center}

Así, verte de lejos, definitivamente.\\
Tu vas con otro hombre, y yo con otra mujer.\\
Y sí que como el agua que brota de una fuente\\
aquellos bellos días ya no pueden volver.
\newline

Así, verte de lejos y pasar sonriente,\\
como quien ya no siente lo que sentía ayer,\\
y lograr que mi rostro se quede indiferente\\
y que el gesto de hastío parezca de placer.
\newline

Así, verte de lejos, y no decirte nada\\
ni con una sonrisa, ni con una mirada,\\
y que nunca sospeches cuánto te quiero así.
\newline

Porque, aunque nadie sabe lo que a nadie le digo,\\
la noche entera es corta para soñar contigo\\
y todo el día es poco para pensar en ti.
\newline

José Ángel Buesa.
\end{verse}

\newpage
\begin{verse}
\begin{center}
\section{Poema del amor pequeño.}
\end{center}
Fue breve aquella noche. Fue breve, pero bella.\\
Poca cosa es el tiempo, que es también poca cosa,\\
porque nadie ha sabido lo que dura una estrella\\
aunque todos sepamos lo que dura una cosa.
\newline

Nuestro amor de una noche fue un gran amor pequeño\\
que rodó por la sombra como un dado sin suerte,\\
pero nadie ha sabido lo que dura un ensueño\\
aunque todos sepamos lo que dura la muerte.
\newline

Una noche es eterna para el que no la olvida,\\
y el tiempo nada importa para el sueño y la flor,\\
y, como nadie sabe lo que dura la vida,\\
nadie sabe tampoco lo que dura el amor.
\newline

José Ángel Buesa.
\end{verse}

\newpage
\begin{verse}
\begin{center}
\section{Poema del poema.}
\end{center}
Quizás pases con otro que te diga el oído\\
esas frases que nadie como yo te dirá;\\
y, ahogando para siempre mi amor inadvertido\\
¡te amaré más que nunca... y jamás lo sabrás! 
\newline

La desolada estrofa, como si fuera un ala,\\
voló sobre el silencio...Y tú estabas allí:\\
Allí en el más oscuro rincón de aquella sala,\\
estabas tú, escuchando mis versos para ti.
\newline

Y tú, la inaccesible mujer de ese poema\\
que ofrece su perfume pero oculta su flor,\\
quizás supiste entonces la amargura suprema\\
de quien ama la vida porque muere de amor.
\newline

Y tú, que nada sabes, que tal vez ni recuerdes\\
aquellos versos tristes y amargos como el mar,\\
cerraste en un suspiro tus grandes ojos verdes,\\
los grandes ojos verdes que nunca he de olvidar.
\newpage

Después, se irguió tu cuerpo como una primavera,\\
mujer hoy y mañana distante como ayer...\\
vi que te alejabas sin sospechar siquiera\\
¡que yo soy aquel hombre... y tú aquella mujer!
\newline

José Ángel Buesa.
\end{verse}

\newpage
\begin{verse}
\begin{center}
\section{Celos.}
\end{center}

Ya solo eres aquella\\
que tiene la costumbre de ser bella.\\
Ya pasó la embriaguez.\\
Pero no olvido aquel deslumbramiento,\\
aquella gloria del primer momento,\\
al ver tus ojos por primera vez.
\newline

Y se que, aunque quisiera,\\
no he de volverte a ver de esa manera.\\
Como aquel instante de embriaguez;\\
y siento celos al pensar que un día,\\
alguien, que no te ha visto todavía,\\
verá tus ojos por primera vez.
\newline

José Ángel Buesa.
\end{verse}

\newpage
\begin{verse}
\begin{center}
\section{Canción para la esposa ajena.}
\end{center}
Tal vez guardes mi libro en alguna gaveta,\\
sin que nadie descubra cual relata su historia,\\
pues será simplemente, los versos de un poeta,\\
tras de arrancar la página de la dedicatoria…
\newline

Y pasarán años... Pero acaso algún día,\\
o acaso alguna noche que estés sola en tu lecho,\\
abrirás la gaveta - como una rebeldía,\\
y leerás mi libro- tal vez como un despecho.
\newline

Y brotará un perfume de una ilusión suprema\\
sobre tu desencanto de esposa abandonada.\\
Y entonces con orgullo, marcaras la página...\\
y guardarás mi libro debajo de la almohada.
\newline

José Ángel Buesa.
\end{verse}

\newpage
\begin{verse}
\begin{center}
\section{No era amor.}
\end{center}
No era amor. Fue otra cosa\\
Pero según murmuran en la ciudad aquella,\\
yo cometí el delito de inventarte una estrella,\\
y fue tuyo el pecado de ofrecerme una rosa.
\newline

No era amor, no era eso\\
que se enciende en la sangre como una llamarada;\\
Era mirar tus ojos y no decirte nada\\
o acercarme a tu boca sin codiciar un beso.
\newline

Tarde para mi hastío,\\
tarde para tu angustia de mariposa en vano,\\
era como dos ciegos que se daban la mano,\\
como dos niños pobres, tu corazón y el mío.
\newline

Nada más. Ni siquiera\\
suspirar en la lluvia de una tarde vacía,\\
No era amor, fue otra cosa. No se lo que sería\\
Yo sé que es triste que nadie lo creyera.
\newline

José Ángel Buesa.
\end{verse}

\newpage
\begin{verse}
\begin{center}
\section{Te acordarás un día.}
\end{center}
Te acordaras un día de aquel amante extraño\\
que te besó en la frente para no hacerte daño.\\
Aquel que iba en la sombra con la mano vacía\\
porque te quiso tanto... que no te lo decía.
\newline

Aquel amante loco... que era como un amigo,\\
y que se fue con otra... para soñar contigo.
\newline

Te acordarás un día de aquel extraño amante.\\
Profesor de horas lentas con alma de estudiante.\\
Aquel hombre lejano... que volvió del olvido\\
solo para quererte... como a nadie ha querido.
\newline

Aquel que fue ceniza de todas las hogueras\\
y te cubrió de rosas sin que tu lo supieras.
\newline

Te acordarás un día del hombre indiferente\\
que en las tardes de lluvia te besaba en la frente.\\
Viajero silencioso de las noches de estío\\
que miraba tus ojos, como quien mira un río.
\newline

Te acordaras un día de aquel hombre lejano\\
del que más te ha querido... porque te quiso en vano.
\newpage

Quizás así de pronto... te acordarás un día\\
de aquel hombre que a veces callaba y sonreía.
\newline

Tu rosal preferido se secará en el huerto\\
como para decirte que aquel hombre se ha muerto.\\
Y él andará en la sombra con su sonrisa triste.\\
Y únicamente entonces sabrás que lo quisiste.
\newline

José Ángel Buesa.
\end{verse}

\newpage
\begin{verse}
\begin{center}
\section{Ya todos la olvidaron.}
\end{center}
Ya todos la olvidaron. Ahora sí que se ha ido,\\
pero, sobre las rosas de la tumba reciente,\\
florecía el recuerdo más allá del olvido…\\
Yo era el hosco, el ausente.
\newline

Qué le importa a la noche que se apague una estrella,\\
si el mar sigue cantando cuando pierde una ola.\\
Ya están secos los ojos que lloraron por ella.\\
Ya se ha quedado sola.
\newline

Ahora ya sigue, sola, su viaje hacia el espanto,\\
por las noches profundas, bajo el cielo inclemente.\\
Ya nadie me reprocha que no lloré aquel llanto,\\
que fui el hosco, el ausente...
\newline

Ya nadie le disputa su silencio y su sombra,\\
sobre todo su sombra, bajo la luz del día.\\
Ya todos la olvidaron, Señor. Nadie la nombra.\\
Yo la recuerdo todavía...
\newline

José Ángel Buesa.
\end{verse}

\newpage
\begin{verse}
\begin{center}
\section{Poema del amor ajeno.}
\end{center}
Puedes irte y no importa, pues te quedas conmigo\\
como queda un perfume donde había una flor.\\
Tú sabes que te quiero, pero no te lo digo;\\
y yo se que eres mía, sin ser mío tu amor.
\newline

La vida nos acerca y la vez nos separa,\\
como el día y la noche en el amanecer...\\
Mi corazón sediento ansía tu agua clara,\\
pero es un agua ajena que no debo beber…
\newline

Por eso puedes irte, porque, aunque no te sigo,\\
nunca te vas del todo, como una cicatriz;\\
y mi alma es como un surco cuando se corta el trigo,\\
pues al perder la espiga retiene la raíz.
\newline

Tú amor es como un río, que parece más hondo,\\
inexplicablemente, cuando el agua se va.\\
Y yo estoy en la orilla, pero mirando al fondo,\\
pues tu amor y la muerte tienen un más allá.
\newpage

Para un deseo así, toda la vida es poca;\\
toda la vida es poca para un ensueño así...\\
Pensando en ti, esta noche, yo besaré otra boca;\\
y tú estarás con otro... ¡pero pensando en mí!
\newline

José Ángel Buesa.
\end{verse}

\newpage
\begin{verse}
\begin{center}
\section{Poema de una calle.}
\end{center}
Amo esta calle, y amo sus tristes casas\\
en las que se entristecen cumpleaños y bodas,\\
porque esta calle triste, se alegra cuando pasas\\
tú, mujer preferida entre todas.
\newline

Amo esta calle acaso porque en ella subsiste\\
no sé qué somnolencia de arrabal provinciano.\\
Pero a veces la odio, porque aunque siempre es triste\\
me parece más triste cuando te espero en vano.
\newline

Y yo bien sé que esta calle nunca podrá ser bella\\
con sus fachadas sucias y sus portales viejos.\\
Pero sé que es distinta cuando pasas por ella\\
y te miro pasar... desde lejos.
\newline

Por eso amo esta calle de soledad y hastío\\
que ensancha sus aceras para alejar las casas.\\
Mientras te espera en vano mi corazón vacío,\\
¡que es una calle triste por donde nunca pasas!
\newline

José Ángel Buesa.
\end{verse}

\newpage
\begin{verse}
\begin{center}
\section{Nocturno IV.}
\end{center}
Así estás todavía de pie bajo la lluvia,\\
bajo la clara lluvia de una noche de invierno.\\
De pie bajo la lluvia me llega tu sonrisa,\\
de pie bajo la lluvia te encuentra mi recuerdo.\\
Siempre he de recordarte de pie bajo la lluvia,\\
con un polvo de estrellas muriendo en tus cabellos\\
y tu voz que nacía del fondo de tus ojos\\
y tus manos cansadas que se iban en el viento\\
y aquel cielo de plomo y el rumor de los árboles\\
y la hoja aquella que te cayó en el seno\\
y el rocío nocturno dormido en tus pestañas\\
y engarzando diamantes en tu vestido negro.
\newline

Así estás todavía lejanamente cerca\\
desde tu lejanía de sombra y de silencio.\\
Mi corazón te llama de pie bajo la lluvia,\\
de pie bajo la lluvia te acercas en el sueño.\\
La vida es tan pequeña que cabe en una noche.\\
Quizá fue que en la sombra me encontré con tu beso\\
y por eso me envuelve, de pie bajo la lluvia,\\
el sabor de tu boca y el olor de tu cuerpo.
\newpage

Si, me has dejado triste porque pienso que acaso\\
ya no estarás conmigo cuando llueva de nuevo.\\
Y no he de verte entonces de pie bajo la lluvia\\
con las manos temblando de frío y de deseo.\\
Pero aunque habrá otras noches cargadas de perfumes\\
y otras mujeres, y otras, a lo largo del tiempo,\\
siempre he de recordarte de pie bajo la lluvia,\\
bajo la lluvia clara de una noche de invierno...
\newline

José Ángel Buesa.
\end{verse}

\newpage
\begin{verse}
\begin{center}
\section{Tercer poema del río.}
\end{center}
El agua del río pasaba indolente,\\
reflejando noches y arrastrando días…\\
Tú, desnuda en la fresca corriente,\\
reías...
\newline

Yo te contemplaba desde la ribera,\\
tendido a la sombra de un árbol sonoro;\\
y resplandecía tu áurea cabellera,\\
desatada en el agua ligera,\\
como un remolino de espuma de oro…
\newline

Y pasaban las nubes errantes,\\
mientras tú te erguías bajo el sol de estío,\\
con los blancos hombros llenos de diamantes,\\
en la rumorosa caricia del río.
\newline

Y tú te reías…\\
Y mirando mis manos vacías,\\
pensé en tantas cosas que ya fueron mías,\\
y que se me han ido, como tú te irás...
\newpage

Y tendí mis brazos hacia la corriente,\\
hacia la corriente cantarina y clara,\\
porque tuve miedo, repentinamente,\\
de que el agua feliz te arrastrara...
\newline

Y ya no reías…\\
bajo el sol de estío,\\
ni resplandecías de oro y de rocío.\\
Y saliste corriendo del río,\\
y llenaste mis manos vacías...
\newline

Y al sentir tu cuerpo tan cerca y tan mío,\\
al vivir en tu amor un instante\\
más allá del placer y del hastío,\\
vi pasar la sombra de una nube errante,\\
de una nube fugaz sobre el río...
\newline

José Ángel Buesa
\end{verse}

\newpage
\begin{verse}
\begin{center}
\section{Poema de la espera.}
\end{center}
Yo sé que tú eres de otro y a pesar de eso espero.\\
Y espero sonriente porque yo sé que un día\\
como en amor, el último vale más que el primero\\
tu tendrás que ser mía.
\newline

Yo sé que tú eres de otro pero eso no me importa.\\
Porque nada es de nadie si hay alguien que lo ansía.\\
Y mi amor es tan largo y la vida es tan corta\\
que tendrás que ser mía.
\newline

Yo sé que tú eres de otro.\\
Pero la sed se sacia solamente en el fondo de la copa vacía.\\
Y como la paciencia puede más que la audacia\\
tu tendrás que ser mía.
\newline

Por eso en lo profundo de mis sueños despiertos\\
yo seguiré esperando porque sé que algún día\\
buscarás el refugio de mis brazos abiertos\\
y tendrás que ser mía.
\newline

José Ángel Buesa.
\end{verse}

\newpage
\begin{verse}
\begin{center}
\section{Canción del viaje.}
\end{center}
Recuerdo un pueblo triste y una noche de frío\\
y las iluminadas ventanillas de un tren.\\
Y aquel tren que partía se llevaba algo mío,\\
ya no recuerdo cuando, ya no recuerdo quien.
\newline

Pero sí que fue un viaje para toda la vida\\
y que el último gesto, fue un gesto de desdén,\\
porque dejó olvidado su amor sin despedida\\
igual que una maleta tirada en el andén.
\newline

Y así, mi amor inútil, con su inútil reproche,\\
se acurrucó en su olvido, que fue inútil también.\\
Como esos pueblos tristes, donde llueve de noche,\\
como esos pueblos tristes, donde no para el tren.
\newline

José Ángel Buesa.
\end{verse}

\newpage
\begin{verse}
\begin{center}
\section{Canción cotidiana.}
\end{center}
Tu amor llegó calladamente;\\
calladamente se me fue...\\
Porque el amor es una fuente\\
que se nos seca de repente,\\
sin saber cómo ni por qué.
\newline

Amor de un beso que se olvida\\
y de un suspiro que se va;\\
amor de paso en nuestra vida,\\
pues se le da la bienvenida\\
cuando tal vez se aleja ya.
\newline

Así tu amor fue como el mío,\\
mujer de un claro atardecer:\\
amor que pasa como un río,\\
sin estancarse en el hastío\\
ni repetirse en el placer.
\newline

Amor feliz que da sin tasa,\\
pues sólo pide, a cambio, amor;\\
amor que deja, cuando pasa,\\
no la ceniza de una brasa,\\
Sino el perfume de una flor.
\newpage

Amor que al irse no está ausente;\\
amor sin dudas y sin fe,\\
como este amor intrascendente,\\
que, si llegó calladamente,\\
calladamente se fue...
\newline

José Ángel Buesa.
\end{verse}

\newpage
\begin{verse}
\begin{center}
\section{Canción de la búsqueda.}
\end{center}
Todavía te busco, mujer que busco en vano,\\
mujer que tantas veces cruzaste mi sendero,\\
sin alcanzarte nunca cuando extendí la mano\\
y sin que me escucharas cuando dije: «te quiero...»
\newline

Y, sin embargo, espero. Y el tiempo pasa y pasa.\\
Y ya llega el otoño, y espero todavía:\\
De lo que fue una hoguera sólo queda una brasa,\\
pero sigo soñando que he de encontrarte un día.
\newline

Y quizás, en la sombra de mi esperanza ciega,\\
si al fin te encuentro un día, me sentiré cobarde,\\
al comprender, de pronto, que lo que nunca llega\\
nos entristece menos que lo que llega tarde.
\newline

Y sentiré en el fondo de mis manos vacías,\\
más allá de la bruma de mis ojos huraños,\\
la ansiedad de las horas convirtiéndose en días\\
y el horror de los días convirtiéndose en años...
\newpage

Pues quizás esté mustia tu frente soñadora,\\
ya sin calor la llama, ya sin fulgor la estrella...\\
Y al no decir: «¡Es ella!» como diría ahora.\\
seguiré mi camino, murmurando: «Era ella...»
\newline

José Ángel Buesa.
\end{verse}

\newpage
\begin{verse}
\begin{center}
\section{Amor tardío.}
\end{center}
Tardíamente, en el jardín sombrío,\\
tardíamente entró una mariposa,\\
transfigurando en alba milagrosa\\
el deprimente anochecer de estío.
\newline

Y, sedienta de miel y de rocío,\\
tardíamente en el rosal se posa,\\
pues ya se deshojó la última rosa\\
con la primera ráfaga de frío.
\newline

Y yo, que voy andando hacia el poniente,\\
siento llegar maravillosamente,\\
como esa mariposa, una ilusión;
\newline

pero en mi otoño de melancolía,\\
mariposa de amor, al fin del día,\\
qué tarde llegas a mi corazón...
\newline

José Ángel Buesa
\end{verse}

\newpage
\begin{verse}
\begin{center}
\section{Canción de los amantes.}
\end{center}
Donde quiera en las noches\\
se abrirá una ventana\\
O una puerta cualquiera\\
de una calle lejana.\\
No importa donde o cuando...\\
puede ser donde quiera\\
Ni menos en otoño,\\
ni más en primavera.
\newline

Y hoy igual que mañana,\\
mañana igual que ayer\\
Un hombre enloquecido\\
besará una mujer.
\newline

Tal vez nadie lo sepa...\\
Como tal vez un día\\
Todos irán sabiendo\\
lo que nadie sabía.
\newpage

Y para los amantes\\
su amor desesperado\\
Podrá ser un delito...\\
pero nunca un pecado.
\newline

Por eso el amor pasa\\
por las calles desiertas\\
Y es como un viento loco\\
que quiere abrir las puertas.
\newline

Bien saben los amantes\\
que hay caricias que son\\
no una simple caricia\\
sino una posesión.\\
Y que un beso... uno solo\\
puede más que el olvido\\
Si se juntan dos bocas\\
en un beso prohibido.
\newpage

No, un gran amor no es grande\\
por lo mucho que dura\\
Si se parece a un árbol\\
reseco en la llanura.\\
Y los amantes saben,\\
que sin querer siquiera\\
Hay un amor que crece\\
como una enredadera.
\newline

Es natural que el agua\\
de un estanque sombrío\\
Sueñe en sus largas noches\\
con el viaje de un río.
\newline

Y si por algo es triste\\
la lluvia que no llueve\\
Será porque es la lluvia\\
condenada a ser nieve.
\newline

Es natural que un día\\
comprendan los amantes\\
que no hay nunca sin siempre...\\
que no hay después sin antes.\\
Y así brota en el alma\\
la rebelión de un sueño\\
Que es como un perro arisco\\
que le gruñe a su dueño.
\newline

El amor... esa estrella\\
de una sombra infinita\\
Aunque muera cien veces...\\
cien veces resucita
\newline

Y suele ser un niño\\
de manos milagrosas\\
Que rompe las cadenas\\
y hace nacer las rosas.
\newline

Ya no habrá días turbios...\\
ya no habrá noches malas\\
Si hay un amor secreto\\
que nos presta sus alas.
\newline

Y el corazón renace\\
con renovada fe\\
Igual que los rosales...\\
que no saben por qué.
\newpage

Donde quiera en las noches,\\
puede abrirse una puerta\\
Pero... tan suavemente\\
que nadie se despierta
\newline

Puede ser en otoño...\\
puede ser en verano\\
Tanto un amor tardío...\\
como un amor temprano.\\
Una mujer... un hombre...\\
y un oscuro aposento\\
Y allá afuera en la calle...\\
sigue pasando el viento.\\
Y si en la noche hay algo\\
queriendo amanecer\\
Es simplemente un hombre\\
que besa a una mujer.
\newline

José Ángel Buesa.
\end{verse}

\newpage
\begin{verse}
\begin{center}
\section{Con la simple palabra.}
\end{center}
Con la simple palabra de hablar todos los días,\\
qué es tan noble que nunca llegará a ser vulgar,\\
voy diciendo estas cosas que casi no son mías,\\
así como las playas casi no son mar.
\newline

Con la simple palabra con que se cuenta un cuento,\\
que es la vejez eterna de la eterna niñez,\\
la ilusión, como un árbol que se deshoja al viento,\\
muere con la esperanza de nacer otra vez.
\newline

Con simple palabra te ofrezco lo que ofreces,\\
amor que apenas llegas cuando te has ido ya:\\
Quien perfuma una rosa se equivoca dos veces,\\
pues la rosa se seca y el perfume se va.
\newline

Con la simple palabra que arde en su propio fuego,\\
siento que en mí es orgullo lo que en otro es desdén:\\
Las estrellas no existen en las noches del ciego,\\
pero, aunque él no lo sepa, lo iluminan también.
\newpage

Y así, como un arroyo que se convierte en río,\\
y que en cada cascada se purifica más,\\
voy cantando este canto tan ajeno y tan mío,\\
con la simple palabra que no muere jamás.
\newline

José Ángel Buesa.
\end{verse}

\newpage
\begin{verse}
\begin{center}
\section{Cuartetos del transeúnte.}
\end{center}
Sonríe, jardinera, si en el surco te inclinas\\
y buscas el secreto profundo de las cosas\\
no pienses que las rosas se afean con espinas;\\
sino que las espinas se embellecen con rosas.
\newline

Jugué al amor contigo, con vanidad tan vana\\
que marqué con la uña los naipes que te di.\\
Y en ese extraño juego, donde pierde el que gana,\\
gané tan tristemente, que te he perdido a ti.
\newline

Al referir mi viaje le fui añadiendo cosas.\\
Cosas que sueño a veces, pero que nunca digo,\\
y así, donde vi un yermo, juré haber visto rosas.\\
No me culpes, muchacha, que igual hice contigo.
\newline

Yo sólo pude recordar tu nombre,\\
tú, en cambio, recordaste cada fecha de ayer.\\
Y aprendí que las cosas que más olvida un hombre,\\
son las cosas que siempre recuerda una mujer.
\newpage

Aquí estaba la hierba, viajero de una hora,\\
y, cuando te hayas ido, seguirá estando aquí.\\
Bien poco ha de importarle que la pises ahora\\
sabiendo que mañana nacerá sobre ti.
\newline

José Ángel Buesa.
\end{verse}

\newpage
\begin{verse}
\begin{center}
\section{Carta a usted.}
\end{center}
Señora; según dicen, ya tiene usted otro amante.\\
Lástima que la prisa nunca sea elegante...\\
Yo sé que no es frecuente que una mujer hermosa\\
se resigne a ser viuda, sin haber sido esposa.
\newline

Y me parece injusto discutirle el derecho\\
de compartir sus penas, sus gozos y su lecho;\\
pero el amor, señora, cuando llega el olvido\\
también tiene el derecho de un final distinguido.
\newline

Perdón, si es que la hiere mi reproche, perdón,\\
aunque sé que la herida no es en el corazón...\\
Y, para perdonarme, piense si hay más despecho\\
en lo que yo le digo que en lo que usted ha hecho;
\newline

pues sepa que una dama con la espalda desnuda,\\
sin luto, en una fiesta, puede ser una viuda,\\
pero no, como tantas, de un difunto señor,\\
sino, para ella sola; viuda de un gran amor.
\newpage

Y nuestro amor, recuerdo, fue un amor diferente,\\
(al menos al principio, ya no, naturalmente).

Usted era el crepúsculo a la orilla del mar,\\
que, según quien la mire, será hermoso o vulgar.\\
Usted era la flor que, según quien la corta,\\
es algo que no muere o algo que no importa.
\newline

O acaso ¿cierta noche de amor y de locura,\\
yo vivía un ensueño... y usted una aventura?\\
Si, usted juró, cien veces, ser para siempre mía:\\
yo besaba sus labios, pero no lo creía...
\newline

Usted sabe, y perdóneme, que en ese juramento\\
influye demasiado la dirección del viento.\\
Por eso no me extraña que ya tenga otro amante,\\
a quien quizás le jure lo mismo en este instante.
\newline

Y como usted, señora, ya aprendió a ser infiel,\\
a mí, así de repente... me da pena por él.
\newline

Sí, es cierto. Alguna noche su puerta estuvo abierta,\\
y yo, en otra ventana me olvidé de su puerta;\\
o una tarde de lluvia se iluminó mi vida\\
mirándome en los ojos de una desconocida;
\newpage

y también es posible que mi amor indolente\\
desdeñara su vaso bebiendo en la corriente.\\
Sin embargo, señora, yo, con sed o sin sed,\\
nunca pensaba en otra si la besaba a usted.
\newline

Perdóneme de nuevo, si le digo estas cosas,\\
pero ni los rosales dan solamente rosas;\\
y no digo esto por usted, ni por mí,\\
sino por los amores que terminan así.
\newline

Pero vea, señora, qué diferencia había\\
entre usted que lloraba y yo; que sonreía,\\
pues nuestro amor concluye con finales diversos:\\
Usted besando a otro; yo, escribiendo estos versos...
\newline

José Ángel Buesa.
\end{verse}

\newpage
\begin{verse}
\begin{center}
\section{Discreto amor.}
\end{center}
Mi viejo corazón toca a una puerta,\\
mi viejo corazón, como un mendigo\\
con el afán de su esperanza incierta\\
pero callando lo que yo no digo.
\newline

Porque la que me hirió sin que lo advierta,\\
la que sólo me ve como un amigo\\
si alguna madrugada está despierta\\
nunca será porque soñó conmigo...
\newline

Y sin embargo, ante la puerta oscura\\
mi corazón, como un mendigo loco\\
va a pedir su limosna de ternura
\newline

Y cerrada otra vez, o al fin abierta,\\
no importa si alguien oye cuando toco,\\
porque nadie sabrá cuál es la puerta.
\newline

José Ángel Buesa.
\end{verse}

\newpage
\begin{verse}
\begin{center}
\section{Era mi amiga.}
\end{center}
Era mi amiga, pero yo la amaba\\
yo la amaba en silencio puramente,\\
y mientras sus amores me contaba\\
yo escuchaba sus frases tristemente.
\newline

Era mi amiga, pero me gustaba\\
y mi afán era verla a cada instante.\\
Nunca supo el amor que yo albergaba\\
porque siempre me hablaba de su amante.
\newline

Era mi amiga para todo el mundo\\
porque a nadie mi amor yo confesaba,\\
pero yo la quería muy profundo\\
y forzosamente me callaba.
\newline

Era mi amiga, y mi cuerpo sentía\\
estremecer si ella me miraba,\\
al oírla junto a mí feliz me hacía\\
más de este amor ella nunca supo nada.\\
y aunque sólo mi amistad yo le ofrecía,\\
era mi amiga, pero yo la amaba.
\newline

José Ángel Buesa.
\end{verse}

\newpage
\begin{verse}
\begin{center}
\section{Amamos porque sí.}
\end{center}
Amamos porque sí, sencillamente\\
porque sí, sin saberlo,\\
como cuando la espiga se levanta,\\
como la lluvia cuando está cayendo,\\
como el viento que pasa y no lo sabe\\
y sin embargo, pasa y es el viento.
\newline

Amamos porque sí, sencillamente\\
porque sí, sin razón y sin remedio,\\
como se seca un pozo,\\
como se empaña a veces un espejo,\\
como una fecha que cambió de día\\
o un nombre que olvidamos en un sueño.
\newline

Amamos porque sí, sencillamente\\
y no importa en qué tiempo,\\
si en un amanecer de primavera\\
o en un lento crepúsculo de invierno,\\
pues si el árbol lozano da más flores\\
son más dulces los frutos de los árboles viejos.
\newpage

Amamos porque sí, sencillamente\\
por un porqué fatal que no sabemos,\\
como el traje de luto para un niño\\
o como las estrellas para un ciego,\\
cómo van hacia abajo las raíces\\
y hacia arriba las ramas\\
con las hojas por dentro.
\newline

Amamos porque sí, sencillamente\\
porque sí, porque es cierto,\\
como un anochecer al mediodía,\\
como una llamarada sobre el hielo,\\
como resucitar estando vivos\\
sólo para morir sin haber muerto.
\newline

Amamos porque sí, sencillamente.\\
Sencillamente, como pasa el viento ...
\newline

José Ángel Buesa.
\end{verse}

\newpage
\begin{verse}
\begin{center}
\section{Dúo de amor.}
\end{center}
En el hondo silencio de la noche serena\\
se dilata un lejano perfume de azucena,\\
y aquí, bajo los dedos de seda de la brisa,\\
mi corazón se ensancha como en una sonrisa...
\newline

Y yo sé que el silencio tiene un ritmo profundo\\
donde palpita un eco del corazón del mundo,\\
un corazón inmenso que late no sé dónde,\\
pero que oye el latido del mío, y me responde...
\newline

El corazón que sientes latir en derredor,\\
es un eco del tuyo, que palpita de amor.\\
El corazón del mundo no es ilusorio: Existe.\\
Pero, para escucharlo, es preciso estar triste;
\newline

triste de esa tristeza que no tiene motivo,\\
en esta lenta muerte del dolor de estar vivo.\\
La vida es un rosal cuando el alma se alegra,\\
pero, cuando está triste, da una cosecha negra.
\newline

El amor es un río de luz entre la sombra,\\
y santifica el labio pecador que lo nombra.\\
Sólo el amor nos salva de esta gran pesadumbre,\\
levantando el abismo para trocarlo en cumbre.
\newline

Sólo el amor nos salva del dolor de la vida,\\
como una flor que nace de una rama caída;\\
pues si la primavera da verdor a la rama,\\
el corazón se llena de aroma, cuando ama.
\newline

Amar es triste a veces, más triste todavía\\
que no amar. El amor no siempre es alegría.\\
Tal vez, por eso mismo, es eterno el amor:\\
porque, al dejarnos tristes, hace dulce el dolor.
\newline

Amar es la tristeza de aprender a morir.\\
Amar es renacer. No amar, es no vivir.\\
El amor es a veces lo mismo que una herida,\\
y esa herida nos duele para toda la vida.
\newline

Si cierras esa herida tu vida queda muerta.\\
Por eso, sonriendo, haz que siempre esté abierta;\\
y si un día ella sola se cierra de repente,\\
tú, con tus propias manos, ábrela nuevamente.
\newline

Desdichada alegría que nace del dolor.\\
De un dolor de la rama también nace la flor.\\
Pero de esa flor efímera, como todas, se mustia,\\
y la rama se queda contraída de angustia.
\newpage

Cada hoja que cae deja el sitio a otra hoja,\\
y así el amor -resumen de toda paradoja-\\
renace en cada muerte con vida duradera;\\
porque decir amor, es decir primavera.
\newline

Primavera del alma, primavera florecida\\
que deja un misterioso perfume en nuestra vida.\\
Primavera del alma, de perpetuo esplendor,\\
que convierte en sonrisa la mueca del dolor.
\newline

Primavera de ensueño que nos traza un camino\\
en la intrincada selva donde acecha el destino.\\
Primavera que canta si el huracán la azota\\
y que da nuevo aliento tras de cada derrota.
\newline

Primavera magnánima, cuyo verdor feliz\\
rejuvenece el árbol seco hasta la raíz...\\
Amor es la ley divina de plenitud humana;\\
dolor que hoy nos agobia y añoramos mañana...
\newline

Eso es amor, y amando, también la vida es eso:\\
¡Dos almas que se duermen a la sombra de beso!
\newline

José Ángel Buesa.
\end{verse}

\newpage
\begin{verse}
\begin{center}
\section{Elegía.}
\end{center}
Golondrina del alba sombría,\\
mariposa del alba radiante:\\
Cuánto puede durar un instante,\\
un instante de noche en el día.
\newline

Yo, que supe ignorar tantas cosas,\\
ahora sé que jamás nos veremos,\\
pues te fuiste, empuñando los remos,\\
en tu barca cubierta de rosas.
\newline

Ahora sé la verdad de la tierra,\\
que florece aunque nadie la labre,\\
y la puerta de luz que se abre\\
si una puerta de sombra se cierra.
\newline

Ahora sé que la noche no miente\\
cuando deja de caer su rocío:\\
Fue un rosal a la orilla de un río,\\
y quizás lo arrastró la corriente...
\newpage

Y te fuiste, luciérnaga loca,\\
golondrina del alba sombría,\\
con el tibio sabor de tu boca\\
¡de tu boca que nunca fue mía!
\newline

José Ángel Buesa.
\end{verse}

\newpage
\begin{verse}
\begin{center}
\section{Elegía para entonces.}
\end{center}
Entonces, todavía tu voz me sabrá a luego.\\
Y todavía y luego y siempre serás otra\\
más allá de ti misma, inaccesiblemente…\\
Y siendo tú el mar íntegro, te buscaré en la ola.
\newline

Entonces, en tus ojos flotará todavía\\
aquella vaga música que rimé con las rosas.\\
Y todavía entonces saldré a escuchar tus ecos\\
a las distancias húmedas de palabras redondas.
\newline

Entonces, todavía te esperaré…En ti misma\\
esperaré el retorno lírico de tu otra.\\
Y aromaré la brisa del bosque con tu nombre\\
y en la arena del páramo sembraré mi voz ronca…
\newline

Y la flor, y la piedra, y el árbol, y el sendero,\\
y la raíz, y el ala, y la luz, y la onda\\
me dirán que te vieron pasar como un perfume\\
envuelta en una trémula túnica de palomas...
\newpage

Y la rosa, y la brisa, y la fuente, y el astro,\\
y el pájaro, y el musgo, y la nube, y la fronda\\
me dirán que pasaste cubierta de rocío\\
entre un emocionado vaivén de mariposas
\newline

Y en lo hondo de tus besos habrá un temblor de ausencia,\\
y besaré en el polvo la huella de tus corzas;\\
fatigaré el oráculo del pétalo sonoro\\
y beberé el narcótico del pétalo sin sombra.\\
Y entonces, todavía tu voz me sabrá a nunca,\\
y todavía y siempre esperaré a tu otra\\
más allá de ti misma, inaccesiblemente…\\
Y, siendo tú el sol íntegro, ¡te buscaré en la aurora!
\newline

José Ángel Buesa.
\end{verse}

\newpage
\begin{verse}
\begin{center}
\section{Elegía nocturna.}
\end{center}
Quién nos hubiera dicho... Que todo acabaría\\
como acaba en la sombra la claridad del día.
\newline

Fuiste como la lluvia cayendo sobre un río\\
para que fuera tuyo... todo lo que era mío.
\newline

Fuiste como una lámpara que se encendió en mi vida,\\
yo la soplé de pronto... Pero siguió encendida.
\newline

Fuiste un río ilusorio cantando en un desierto\\
y floreció la arena como si fuera cierto.
\newline

Mi amor fue una gaviota que construyó su nido\\
en lo alto de un mástil... Ahora el buque se ha ido.
\newline

Ahora me envuelve un hosco silencio de campana\\
donde sólo resuena tu campana lejana.
\newline

Y como un surco amargo... Que se negara al trigo\\
ahora mi alma no sueña... Por no soñar contigo.
\newline

José Ángel Buesa.
\end{verse}

\newpage
\begin{verse}
\begin{center}
\section{Elegía por nosotros.}
\end{center}
Erguida en tu silencio y en tu orgullo,\\
no sé con qué señor que te enamora,\\
comentas a manera de murmullo:\\
¡Mirad ese es el hombre que me adora!
\newline

Yo paso como siempre, absorto... mudo,\\
y tú nerviosamente te sonríes,\\
sabiendo que detrás de mi saludo,\\
te ahondas y después te me deslíes.
\newline

Yo sé que ni te busco, ni te sigo,\\
que nada te mendigo, ni reclamo,\\
comento, nada más con un amigo:\\
``Esa es la mujer que yo más amo”.
\newline

Yo sé que mi cariño recriminas,\\
es claro tú no entiendes de esas cosas,\\
qué sabe del perfume y las espinas,\\
quien nunca estuvo al lado de las rosas.
\newline

Tú sabes que jamás suplico nada,\\
y me sabes cautivo de tus huellas,\\
que vivo en la región de tu mirada,\\
y comparto contigo las estrellas.
\newpage
 
Un día nos veremos nuevamente,\\
y es lógico que bajes la cabeza,\\
tendrás muchas arrugas en la frente,\\
y el rostro entristecido y sin belleza.
\newline

Serás menos sensual en la cadera,\\
tus ojos no tendrán aquel hechizo,\\
y aún murmuraré—¡Si me quisiera!\\
tú sólo pensarás: ¡Cuánto me quiso!
\newline

José Ángel Buesa.
\end{verse}

\newpage
\begin{verse}
\begin{center}
\section{El gran amor.}
\end{center}
Un gran amor, un gran amor lejano\\
es algo así como la enredadera\\
que no quisiera florecer en vano\\
y sigue floreciendo aunque no quiera.
\newline

Un gran amor se nos acaba un día\\
y es tristemente igual a un pozo seco,\\
pues ya no tiene el agua que tenía\\
pero le queda todavía el eco.
\newline

Y, en ese gran amor, aquel que ama\\
compartirá el destino de la hoguera,\\
que lo consume todo con su llama\\
porque no sabe arder de otra manera.
\newline

José Ángel Buesa.
\end{verse}

\newpage
\begin{verse}
\begin{center}
\section{El pequeño dolor.}
\end{center}
Mi dolor es pequeño,\\
pero aún así bendigo este dolor,\\
que es como no soñar después de un sueño,\\
o es como abrir un libro y encontrar una flor.
\newline

Déjame que bendiga\\
mi pequeño dolor,\\
que no sabe crecer como la espiga,\\
porque la espiga crece sin amor.
\newline

Y déjame cuidar como una rosa\\
este dolor que nace porque sí,\\
este dolor pequeño, que es la única cosa\\
que me queda de ti.
\newline

José Ángel Buesa.
\end{verse}

\newpage
\begin{verse}
\begin{center}
\section{Inesperadamente.}
\end{center}
Inesperadamente tu amor llega a mi vida,\\
mujer de besos hondos y plenitud creciente,\\
como brota un retoño de una rama caída,\\
como en un río seco renace la corriente.
\newline

Llegas como las nubes, inesperadamente;\\
inesperadamente llegas como el verano,\\
para dejarme el peso de una sombra en la frente\\
y un dolor de raíces profundas en las manos.
\newline

Y es que tu boca alegre me inspira un beso triste,\\
y en tus ojos cercanos veo un mirar ausente,\\
porque sé que algún día, lo mismo que viniste,\\
te me irás de los brazos, inesperadamente...
\newline

José Ángel Buesa.
\end{verse}

\newpage
\begin{verse}
\begin{center}
\section{Madrigal triste.}
\end{center}
¡Qué clara la mañana! ¡qué fresco y delicioso\\
el viento! ¡Cuánta luz! ¡Cuánta leve armonía!...\\
—Busqué a mí alrededor algo maravilloso...\\
Y ella, a mi lado, sonreía...
\newline

¡Cuánta muda tristeza en el cielo nublado!\\
¡Qué silencio en las frondas donde el ave cantaba!\\
—Busqué a mí alrededor algo desconsolado...\\
Y ella, a mi lado, suspiraba...
\newline

¡Qué soledad! ¡Qué angustia crispada en la doliente\\
neblina! ¡Qué vacío en todo!...-Desolado\\
Busqué a mí alrededor... Y busqué inútilmente:\\
Ella no estaba ya a mi lado...
\newline

José Ángel Buesa
\end{verse}

\newpage
\begin{verse}
\begin{center}
\section{Mía.}
\end{center}
Mujer soñada: Ya tú eres mía...\\
Ya tú eres mía, como las rosas\\
son del rosal, y el Sol, del día...\\
Todos los seres, todas las cosas,\\
me están diciendo que ya eres mía...
\newline

¿No oyes el canto que alza el jilguero,\\
revoleteando sobre el alero,\\
vertiendo a chorros su melodía?\\
Es que él bien sabe cuanto te quiero;\\
es porque sabe que ya eres mía...
\newline

¿No sientes cómo la mano blonda\\
del Sol oculto tras de la fronda\\
te unge del oro tibio del día?\\
Es que el Sol sabe también cuán honda,\\
cuán dulcemente ya tú eres mía...
\newline

¿No ves la lluvia —que canta ahora—,\\
regando perlas? Ya ella no llora\\
con infinita melancolía,\\
y es que la lluvia tampoco ignora\\
que ya eres mía...
\newpage

¿No ves los juegos que entre las rocas\\
 las mariposas juegan airosas,\\
en una móvil policromía?\\
Es porque saben las mariposas\\
que ya eres mía...
\newline

¿No estas sintiendo que dulcemente\\
la fresca brisa besa tu frente\\
y alarga el beso sobre la mía?\\
Es que ella sabe cuán hondamente\\
ya tú eres mía...
\newline

¿No ves las noches ahora más bellas?\\
Es que han surgido nuevas estrellas,\\
y entre relámpagos de pedrería,\\
decir parecen que saben ellas\\
que ya eres mía...
\newline

¿No oyes al río, que descendiendo\\
por los barrancos, calma su estruendo\\
y se hace ahora blanda armonía?\\
¿No te parece que va diciendo\\
que ya eres mía?
\newpage

Mujer soñada: Ya tú eres mía,\\
ya tú eres mía como las rosas\\
son del rosal, y el Sol del día.\\
Todos los seres, todas las cosas,
\newline

—ríos, estrellas y mariposas—,\\
oyen el himno de mi alegría,\\
y hay más perfumes, porque hay más rosas,\\
desde que puedo llamarte mía...
\newline

José Ángel Buesa.
\end{verse}

\newpage
\begin{verse}
\begin{center}
\section{Nocturno VII.}
\end{center}
Ahora que ya te fuiste, te diré que te quiero.\\
Ahora que no me oyes, ya no debo callar.\\
Tú seguirás tu vida y olvidarás primero...\\
Y yo aquí, recordándote, a la orilla del mar.
\newline
 
Hay un amor tranquilo que dura hasta la muerte,\\
y un amor tempestuoso que no puede durar.\\
Acaso aquella noche no quise retenerte...\\
y ahora estoy recordándote a la orilla del mar.
\newline

Tú, que nunca supiste lo que yo te quería,\\
quizás entre otros brazos lograrás olvidar...\\
Tal vez mires a otro, igual que a mí aquel día...\\
Y yo aquí, recordándote, a la orilla del mar.
\newline

El rumor de mi sangre va cantando tu nombre,\\
y el viento de la noche lo repite al pasar.\\
Quizás en este instante tú besas a otro hombre...\\
Y yo aquí, recordándote, a la orilla del mar...\\
Y yo aquí, recordándote, a la orilla del mar...
\newline

José Ángel Buesa.
\end{verse}

\newpage
\begin{verse}
\begin{center}
\section{Lied.}
\end{center}
Mi corazón se queda aunque mi amor se vaya,\\
porque el recuerdo nace de un ansia de olvidar.\\
Tu amor tiene la tibia ternura de una playa;\\
mi amor es inestable como el viento y el mar.
\newline

Aunque mi amor se vaya no has de quedarte sola,\\
pues te dejo el reflejo de la luz que encendí:\\
Tu amor es una playa, mi amor es una ola,\\
y necesariamente yo he de volver a ti...
\newline

José Ángel Buesa.
\end{verse}

\newpage
\begin{verse}
\begin{center}
\section{Madrigal de la lluvia de abril.}
\end{center}
Ya no sé bien el sitio ni la hora,\\
ni por qué fuiste mía, ni por qué te perdí.\\
Sé que llovía como llueve ahora,\\
aunque ahora es más triste porque llueve sin ti.
\newline

Y sé que, de repente, cayeron dos diamantes\\
sobre tus zapaticos de charol...\\
Y era dulce aquel llanto de tus ojos radiantes,\\
como esos mediodías en que llueve con sol.
\newline

José Ángel Buesa.
\end{verse}

\newpage
\begin{verse}
\begin{center}
\section{La fuga infinita.}
\end{center}
Se fue mi niñez...
Batiendo sus alas de rosa partió...\\
Le rogué, llorando: «¡Vuelve a mí otra vez!»\\
—Volveré— me dijo... Pero no volvió...
\newline

Después, mi inocencia, cual mística flor,\\
se mustió entre las\\
llamaradas locas del pagano amor,\\
y a mi alma su aroma no tornó jamás...
\newline

Y, al llegar mis dudas, se marchó mi fe...\\
—«¿Volverás?»— le dije... No sé si me oyó:\\
Hizo un gesto vago me miró y se fue.
\newline

Luego, acurrucada, sufrió mi ilusión\\
de los desengaños el flagelo cruel:\\
Me miró con húmedos ojos de lebrel\\
y se fue en silencio de mi corazón...
\newline

Y yo sé que un día también tú te irás,\\
sin que mis caricias puedan retenerte,\\
pues ya hacia otros brazos, o ya hacia la muerte,\\
no te detendrás...
\newpage

Porque sé que un día llegará el olvido,\\
y sé que ese día te me irás, mujer,\\
como tantas cosas que ya se me han ido:\\
¡Para no volver!...
\newline

José Ángel Buesa.
\end{verse}

\newpage
\begin{verse}
\begin{center}
\section{Tercer poema de la despedida.}
\end{center}
Llamarada de ayer, ceniza ahora,\\
ya todo será en vano,\\
cómo fijar el tiempo en una hora\\
o retener el agua en una mano.
\newline

Ah, pobre amor tardío,\\
es tu sombra no más lo que regresa,\\
porque si el vaso se quedó vacío\\
nada importa que esté sobre la mesa.
\newline

Pero quizás mañana,\\
como este gran olvido es tan pequeño,\\
pensaré en ti, cerrando una ventana,\\
abriendo un libro o recordando un sueño...
\newline

Tu amor ya está en mi olvido,\\
pues, como un árbol en la primavera,\\
si florece después de haber caído,\\
no retoña después de ser hoguera;
\newpage

pero el alma vacía\\
se complace evocando horas felices,\\
porque el árbol da sombra todavía,\\
después que se han secado sus raíces;
\newline

y una ternura nueva\\
me irá naciendo, como el pan del trigo:\\
Pensar en ti una tarde, cuando llueva,\\
o hacer un gesto que aprendí contigo.
\newline

Y un día indiferente,\\
ya en olvido total sobre mi vida,\\
recordaré tus ojos de repente,\\
viendo pasar a una desconocida...
\newline

José Ángel Buesa.
\end{verse}

\newpage
\begin{verse}
\begin{center}
\section{Nocturno VIII.}
\end{center}
Aquí, solo en la noche, ya es posible la muerte.\\
Morir es poca cosa si tu amor está lejos.
\newline

Puedo cerrar los ojos y apagar las estrellas.\\
Puedo cerrar los ojos y pensar que ya he muerto.
\newline

Puedo matar tu nombre pensando que no existes.\\
Ahora, solo en la noche, sé que todo lo puedo.
\newline

Puedo extender los brazos y morir en la sombra,\\
y sentir el tamaño del mundo en mi silencio.
\newline

Puedo cruzar los brazos mirándote desnuda,\\
y navegar por ríos que nacen en tu sueño.
\newpage

Sé que todo lo puedo porque la noche es mía,\\
la gran noche que tiembla de un extraño deseo.
\newline

Sé que todo lo puedo, porque puedo olvidarte:\\
Sí. En esta sombra, solo, sé que todo lo puedo.
\newline

Y ya ves: me contento con cerrar bien los ojos\\
y apagar las estrellas y pensar que me he muerto.
\newline

José Ángel Buesa
\end{verse}

\newpage
\begin{verse}
\begin{center}
\section{Oasis.}
\end{center}
Así como un verdor en el desierto,\\
con sombra de palmeras y agua caritativa,\\
quizás ser tu amor lo que me sobreviva,\\
viviendo en un poema después que yo haya muerto.
\newline

En ese canto, cada vez más mío,\\
voces indiferentes repetirán mi pena,\\
y tú has de ser entonces como un rastro en la arena,\\
casi como una nube que pasas sobre un río...
\newline

Tú serás para todos una desconocida,\\
tú que nunca sabrás cómo he sabido amarte;\\
y alguien, tal vez, te buscará en mi arte,\\
y al no hallarte en mi arte, te buscará en mi vida.
\newline

Pero tú no estarás en las mujeres\\
que alegraron un día mi tristeza de hombre:\\
Como oculté mi amor sabré ocultar tu nombre,\\
y, al decir que te amo, nunca diré quién eres.
\newline

Y dirán que era falsa mi pasión verdadera,\\
que fue sólo un ensueño la mujer que amé tanto;\\
o dirán que era otra la que canté en mi canto,\\
otra, que nunca amé ni conocí siquiera.
\newpage

Y así será mi gloria lo que fue mi castigo,\\
porque, como un verdor en el desierto,\\
tu amor me hará vivir después que yo haya muerto,\\
pero cuando yo muera, ¡tú morirás conmigo!
\newline

José Ángel Buesa.
\end{verse}

\newpage
\begin{verse}
\begin{center}
\section{Poema de la desposada.}
\end{center}
Buena suerte, muchacha. Lucirás muy bonita\\
con el velo de novia y el ramo de azahar,\\
pero sin el sonrojo de la primera cita,\\
sino pálida y seria delante del altar.
\newline

Pronto será la boda. Pero acaso un despecho,\\
amargará las noches de tu luna de miel,\\
si al abrir una puerta reconoces un lecho\\
o al cruzar un pasillo recuerdas otro hotel.
\newline

Sin embargo, muchacha, cuando termine el viaje,\\
ya serás la señora de no sé qué señor,\\
aunque tal vez descubras, al abrir tu equipaje,\\
que en la prisa, ¡qué pena!, se te olvidó el amor.
\newline

José Ángel Buesa.
\end{verse}

\newpage
\begin{verse}
\begin{center}
\section{Poema del espejo.}
\end{center}
Déjame ser tu espejo, supliqué aquel día,\\
recuerdo que tu mano se estremeció en la mía.\\
Yo que envidio tu espejo, quiero saber que siente\\
al copiar en la alcoba tu cuerpo adolescente.
\newline

Detrás de los almendros, casi como del fondo\\
del mar, surgió la luna, con su espejo redondo.\\
Te vi de pie en la sombra, junto al lecho vacío\\
se oyó un rumor de sedas, como el rumor de un río.
\newline

Y yo, como el espejo de aquella alcoba oscura,\\
yo, allí solo contigo, reflejé tu hermosura.\\
Fue un instante, en la sombra. No sé bien todavía\\
si eras tú, si fue un sueño, o una flor que se abría.
\newline

Muchacha de la noche de un día diferente,\\
yo no envidio a tu espejo, ya sé que nada siente,\\
Ya sé que te duplica sin comprender siquiera\\
que eres mujer, y hermosa como la primavera,
\newline

Pues si lo comprendiera saltaría en pedazos\\
por el ansia imposible de tenderte los brazos.
\newline

José Ángel Buesa.
\end{verse}

\newpage
\begin{verse}
\begin{center}
\section{Poema del éxtasis.}
\end{center}
No... nunca fue mi mano más lenta que en la hora\\
Secretamente mía de aquella noche, aquella...\\
Fue así como una nube cuando oculta una estrella\\
O así como una estrella que se pierde en la aurora.
\newline

Nunca tuvo mi mano más quietud impaciente,\\
Semejante a la mano de un ladrón inexperto.\\
Porque fue como un buque que oscilara en el puerto\\
Con el ansia inconforme de zarpar de repente.
\newline

Si, aquella noche... noche para soñar en vano\\
O encender una estrella... O apagar una duda.\\
Surgió bajo mi mano tu belleza desnuda\\
Como si tu belleza surgiera de mi mano.
\newline

Ni una sola palabra de temor o reproche\\
Abrevió el retardado placer del desenlace.\\
Cómo crece un jacinto frente al alba que nace\\
O cómo nace el alba del fondo de la noche.
\newline

No... nunca fue una mano más lenta ni más leve\\
Que mi mano de amante con su gesto de amigo.\\
Eras como la nieve cayendo sobre el trigo\\
O un trigo milagroso brotando de la nieve.
\newpage

Y tu estabas inmóvil bajo la selva rosa\\
Como una flor fantástica que se abriera en el lecho.\\
Mientras mi mano lenta descubría en tu pecho\\
Dos motivos iguales para llamarte hermosa.
\newline

Pero desde esa noche de calma y de tormenta\\
Desorientadamente vacilé en una duda.\\
Si cerraste los ojos por no verte desnuda\\
O bien porque mi mano fue demasiado lenta.
\newline

José Ángel Buesa.
\end{verse}

\newpage
\begin{verse}
\begin{center}
\section{Poema final por nosotros.}
\end{center}
Está bien, vas con otro, y me apeno y sonrío,\\
pues recuerdo las noches que temblaste en mi mano,\\
como tiembla en la hoja la humedad del rocío,\\
o el fulgor de la estrella que desciende al pantano.
\newline

Te perdono, y es poco. Te perdono, y es todo,\\
yo que amaba tus formas, más amaba tu amor,\\
y empezó siendo rosa lo que luego fue lodo,\\
a pesar del perfume y a pesar del color.
\newline

Hoy prefiero mil veces sonreír aunque pierda,\\
mientras pierda tan solo el derecho a tu abrazo,\\
y no ser el que olvida, mientras él quien recuerda,\\
y tú bajes el rostro y él lo vuelva si paso.
\newline

Quien te lleva no sabe que pasó mi tormento,\\
y me apena su modo de aferrarse a lo vano,\\
él se aferra a la rosa, pero olvida que el viento,\\
todavía dirige su perfume a mi mano.
\newline

Y por ser quien conozco tus angustias y anhelos,\\
te perdono si pasas y si no me saludas,\\
pues prefiero el orgullo de perderte con celos,\\
a la angustia que él siente de tenerte con dudas.
\newpage

Y mañana quien sabe, no sabré si fue rubia,\\
si canela, o si blanca la humedad de esta pena,\\
y quizás te recuerde si me adentro en la lluvia,\\
o tal vez me dé risa si acaricio la arena.
\newline

José Ángel Buesa.
\end{verse}

\newpage
\begin{verse}
\begin{center}
\section{Aniversario.}
\end{center}
Hoy hace un año, justamente un año.\\
Y llueve como entonces en el atardecer.\\
Y es una lluvia lenta, tan lenta que hace daño,\\
porque casi no llueve ni deja de llover.
\newline

Mi pena es una pena sin tamaño,\\
en el tamaño triste de un nombre de mujer,\\
aunque la gente pasa sin saber qué hace un año,\\
y aunque la lluvia ignora que llueve como ayer...
\newline

José Ángel Buesa.
\end{verse}

\newpage
\begin{verse}
\begin{center}
\section{Triste es saber.}
\end{center}
Triste es saber que nuestra vida\\
es sólo interminable adiós\\
que, como un cuervo trágico,\\
aletea en nuestro corazón;\\
que cada paso nuestro, deja algo\\
más que una huella en pos,\\
algo que ya no vuelve a nuestra vida,\\
que para siempre huyó;\\
que lo que es hoy sonora melodía\\
o encantada canción,\\
será mañana cual rumor de hojas\\
que el viento sacudió...
\newline

Y en esta hora de melancolía,\\
sufro el hondo dolor\\
de preguntarme inútilmente,\\
cuánto me durará tu amor...\\
Que yo bien sé que cual la brisa\\
deja sin perfume a la flor;\\
que como el mar al fin borra\\
la estela que un buque le dejó;\\
que cual se desvanecen los colores\\
de las flores, al sol,\\
y que como la alquimia del otoño\\
trueca en oro el verdor,\\
el nuestro en nuestras vidas obra el paso\\
igual transformación,\\
dejando despertares donde sueños\\
y hastío donde amor...
\newline

Y tengo mucho miedo de esa hora\\
que puede sonar hoy,\\
cuando al besar tus labios, sólo el frío\\
responda a mi calor...\\
Y yo tengo mucho miedo de ese hastío\\
que puedo sentir yo.\\
que robará a mis ojos el miraje\\
azul de la ilusión...
\newline

Y, en esta hora de melancolía,\\
sufro el agrio dolor\\
de no ignorar que un día, quizás pronto,\\
nos diremos adiós...
\newline

José Ángel Buesa.
\end{verse}

\newpage
\begin{verse}
\begin{center}
\section{Amémonos.}
\end{center}
Buscaba mi alma con afán tu alma,\\
buscaba yo la virgen que mi frente\\
tocaba con su labio dulcemente\\
en el febril insomnio del amor.
\newline

Buscaba la mujer pálida y bella\\
que en sueño me visita desde niño,\\
para partir con ella mi cariño,\\
para partir con ella mi dolor.
\newline

Como en la sacra soledad del templo\\
sin ver a Dios se siente su presencia,\\
yo presentí en el mundo tu existencia,\\
y, como a Dios, sin verte, te adoré.
\newline

Y demandando sin cesar al cielo\\
la dulce compañera de mi suerte,\\
muy lejos yo de ti, sin conocerte\\
en la ara de mi amor te levanté.
\newline

No preguntaba ni sabía tu nombre,\\
¿En dónde iba a encontrarte? lo ignoraba;\\
pero tu imagen dentro el alma estaba,\\
más bien presentimiento que ilusión.
\newpage

Y apenas te miré... tú eras ángel\\
compañero ideal de mi desvelo,\\
la casta virgen de mirar de cielo\\
y de la frente pálida de amor.
\newline

Y a la primera vez que nuestros ojos\\
sus miradas magnéticas cruzaron,\\
sin buscarse, las manos se encontraron\\
y nos dijimos ``te amo" sin hablar.
\newline

Un sonrojo purísimo en tu frente,\\
algo de palidez sobre la mía,\\
y una sonrisa que hasta Dios subía...\\
así nos comprendimos... nada más.
\newline

¡Amémonos, mi bien! En este mundo\\
donde lágrimas tantas se derraman,\\
las que vierten quizá los que se aman\\
tienen yo no sé qué de bendición.
\newpage

Dos corazones en dichoso vuelo;\\
¡Amémonos, mi bien! Tiendan sus alas\\
amar es ver el entreabierto cielo\\
y levantar el alma en asunción.
\newline

Amar es empapar el pensamiento\\
en la fragancia del Edén perdido;\\
amar es... amar es llevar herido\\
con un dardo celeste el corazón.
\newline

Es tocar los dinteles de la gloria,\\
es ver tus ojos, escuchar tu acento,\\
en el alma sentir el firmamento\\
y morir a tus pies de adoración.
\newline

José Ángel Buesa.
\end{verse}

\newpage
\begin{verse}
\begin{center}
\section{Soneto del tiempo.}
\end{center}
Me verás sonreír, amiga mía,\\
con aquel gesto frívolo de antaño,\\
y hay un viejo dolor que me hace daño,\\
un dolor que me duele todavía.
\newline

Porque no en vano pasan día y día,\\
y día a día llegan año y año,\\
y el júbilo de ayer se queda huraño\\
de soledad y de melancolía.
\newline

No te engañes, amiga, con mi engaño:\\
la copa en que bebiste está vacía,\\
y el oro de sus bordes se hizo estaño;
\newline

y esta frágil corteza de alegría\\
cubre un viejo dolor que me hace daño,\\
un dolor que me duele todavía...
\newline

José Ángel Buesa.
\end{verse}

\newpage
\begin{verse}
\begin{center}
\section{Soneto para un reproche.}
\end{center}
Yo no sé si tú esperas todavía,\\
el gran amor con que soñaste en vano,\\
que era un pozo en la tarde de verano,\\
y era la sed que el pozo calmaría.
\newline

Yo sólo sé que estuvo cerca un día,\\
cuando tú lo creíste más lejano,\\
y fue una llama que se heló en tu mano,\\
al separar tu mano de la mía.
\newline

Así fue: Poca cosa en el olvido,\\
como el viento que llega y ya se ha ido\\
o la rama partida sin dar flor;
\newline

pero no es culpa mía si tú hiciste\\
una cosa vulgar, pequeña y triste,\\
de lo que pudo ser un gran amor.
\newline

José Ángel Buesa.
\end{verse}

\newpage
\begin{verse}
\begin{center}
\section{Segundo poema de la espera.}
\end{center}
Por un agua de hastío voy moviendo estos remos,\\
que pasan tanto al irme y tan poco al volver;\\
pero quizá un día no nos separaremos,\\
mujer mía y ajena, como el amanecer.
\newline

No importa que me quede ni importa que me vaya,\\
mientras pasan las nubes sin dejar de pasar,\\
porque tu corazón es igual que una playa,\\
que, pudiendo ser tierra, nunca llega a ser mar.
\newline

Tu amor nunca responde cuando mi amor te nombra;\\
tu amor, que sin ser mío, tantas veces perdí;\\
y yo empuño los remos y viajo hacia las sombras,\\
pues todo se hace sombra si estoy lejos de ti.
\newline

Filibustero loco tras el botín de un beso,\\
viajo por aguas tristes que me entristecen más;\\
pero tu amor es siempre camino de regreso,\\
mujer que nunca llegas y que nunca te vas.
\newline

Tu amor es un remoto país desconocido,\\
más allá del mañana, más allá del ayer;\\
y ya sólo recuerdo las veces que me he ido\\
recordando las veces que tuve que volver.
\newpage

Hay virtudes tan tristes, que es mejor ser culpable,\\
y más si es una culpa de amor amarte así;\\
pero, si en nuestras vidas hay algo inevitable,\\
inevitable tú serás para mí.
\newline

Ya me duelen las manos de remar en mi hastío;\\
pero yo sé que un día dejaré de remar,\\
y he de mirar el mundo como si fuera mío,\\
y romperé los remos en la orilla del mar...
\newline

José Ángel Buesa.
\end{verse}

\newpage
\begin{verse}
\begin{center}
\section{Soneto en la alcoba.}
\end{center}
Te miraba acostada con mis ojos de bueno,\\
tus ojos aprendían lentamente a soñar,\\
y tu sueño iba a otro, a tu amor en estreno,\\
embriagado de fuga, de capricho y de azar.
\newline

Me tomaste una mano para palpar tu seno,\\
tu corazón latía con el mío a la par:\\
el tuyo acelerado por un amor ajeno,\\
mi corazón tan cerca, sin poderlo alcanzar.
\newline

Así dejé de amarte y empecé a comprenderte.\\
Sentí que me tocaba como un roce de muerte,\\
un dolor voluptuoso, pasajero y vulgar.
\newline

Y mientras me veías mansamente a tu lado,\\
yo escapaba en silencio, para siempre alejado.\\
¡Aunque esta misma noche te vuelva a desnudar!
\newline

José Ángel Buesa
\end{verse}

\newpage
\begin{verse}
\begin{center}
\section{Soneto para la lluvia.}
\end{center}
Mi corazón no sabe lo que espera,\\
pero yo sé que espera todavía,\\
igual que aquella noche que llovía\\
y te besé bajo la enredadera.
\newline

Tu amor se fue como si no se fuera,\\
pues algo tuyo vuelve cada día,\\
y me dejaste la melancolía\\
de doblar el pañuelo a tu manera.
\newline

Esta noche de viento y lluvia fría\\
quiero pensar que, si tu amor volviera,\\
al dejar de llover ya no se iría.
\newline

Y estoy aquí, bajo la enredadera;\\
y, como aquella noche que llovía,\\
mi corazón no sabe lo que espera...
\newline

José Ángel Buesa.
\end{verse}
\newpage
\begin{verse}
\begin{center}
\section{Poema lejano.}
\end{center}
A veces me pregunto dónde estarás ahora,\\
después de tantas noches sin tu mano en la mía,\\
noches de abrir un libro para esperar la aurora,\\
noches de largo viento por la calle vacía.
\newline

A veces me pregunto si hay alguien que te espera,\\
alguien que no conoces, que pasa y te saluda\\
y, como siempre vistes de negro en primavera,\\
no sé si tus vecinas pensarán que eres viuda.
\newline

A veces me imagino como serán las cosas\\
que te son familiares: tu jardín, tu ventana,\\
el búcaro en la mesa para poner las rosas\\
y un desayuno sin mí cada mañana.
\newline

O me quedo pensando qué sentirás tan lejos,\\
en las tardes heladas, al quitarte el abrigo;\\
y cuando vas de compras sin mirar los espejos\\
para que no te digan que ya no voy contigo.
\newline

Y también me pregunto si alguna madrugada\\
prefieres no dormirte para soñar despierta,\\
o cómo se entristece de lluvia tu mirada\\
cuando pasa el cartero sin tocar en tu puerta.
\newpage

Pero no me pregunto si olvidarás mi nombre,\\
ni lo que tú me diste, ni lo que yo te di,\\
pues si te ven un día del brazo de otro hombre\\
tendrá que ser un hombre que se parece a mí…
\newline

José Ángel Buesa.
\end{verse}

\newpage
\begin{verse}
\begin{center}
\section{La mujer aquella.}
\end{center}
A veces me pregunto: «¿Qué habrá sido\\
de la mujer aquella?» Y su mirada\\
me llega desde el fondo del olvido,\\
y oigo su voz, sin que me diga nada.
\newline

Y voy con ella, como en otro mundo,\\
en otro tiempo, nuevamente mía;\\
y es ella, y de repente la confundo\\
con no sé quién, ni dónde, ni qué día.
\newline

Y se me pierde en una calle triste\\
que no recuerdo ya, pero que existe,\\
y allí le digo adiós y no la sigo;
\newline

Porque quizás, a la mujer aquella,\\
si piensa en mí le ocurrirá conmigo\\
lo que me ocurre a mí si pienso en ella...
\newline

José Ángel Buesa.
\end{verse}


\newpage
\begin{verse}
\begin{center}
\section{Lamentaciones de otoño.}
\end{center}
\begin{center}
I
\end{center}

Como tantas cosas lejanas\\
que se acercan sin un rumor,\\
llegaron las primeras canas\\
y quizás el último amor.
\newline

El amor que pasó deprisa,\\
y el que nunca llegó a pasar,\\
entristecieron mi sonrisa\\
igual que un ciego frente al mar.
\newline

Yo soñaba con un cariño\\
que acaso tuve y se me fue,\\
y me eché a llorar como un niño\\
que llora sin saber por qué.
\newline

Hoy asoman rostros extraños\\
sobriamente frente a mí:\\
Hoy llegan los años huraños\\
diciéndome: «Estamos aquí».
\newpage
\begin{center}
II
\end{center}

Y he de morir soñando cosas\\
que deseé y no conseguí...\\
Y seguirán naciendo rosas,\\
pero no serán para mí.
\newline

Yo buscaba las cosas bellas\\
sin importarme en qué lugar.\\
Y otros mirarán las estrellas\\
que yo no volveré a mirar.
\newline

Y nombrar lo que no se nombra\\
un gran silencio y una cruz,\\
y penetrar en esa sombra,\\
yo, que he amado tanto la luz.

\begin{center}
III
\end{center}

Tanto sueños que ya se han ido\\
y que jamás han de volver...\\
Empezar a morir de olvido,\\
¡oh, noche sin amanecer!
\newpage

Apasionadas noches locas,\\
indeciblemente sin par...\\
Pero otros besarán las bocas\\
que yo dejaré de besar.
\newline

Agridulce sabor del beso,\\
áurea isla sin latitud:\\
Aunque sólo sea por eso,\\
no te me vayas, ¡Juventud!
\newline

No te me vayas todavía,\\
porque no me quiero quedar\\
triste de ensueño y de armonía,\\
igual que un ciego frente al mar.
\newline

José Ángel Buesa.
\end{verse}

\newpage
\begin{verse}
\begin{center}
\section{Poema del fracaso.}
\end{center}
Mi corazón, un día, tuvo un ansia suprema,\\
que aún hoy lo embriaga cual lo embriagara ayer;\\
Quería aprisionar un alma en un poema,\\
y que viviera siempre... Pero no pudo ser.
\newline

Mi corazón, un día, silenció su latido,\\
y en plena lozanía se sintió envejecer;\\
Quiso amar un recuerdo más fuerte que el olvido\\
y morir recordando... Pero no pudo ser.
\newline

Mi corazón, un día, soñó un sueño sonoro,\\
en un fugaz anhelo de gloria y de poder;\\
Subió la escalinata de un palacio de oro\\
y quiso abrir las puertas... Pero no pudo ser.
\newline

Mi corazón, un día, se convirtió en hoguera,\\
por vivir plenamente la fiebre del placer;\\
Ansiaba el goce nuevo de una emoción cualquiera,\\
un goce para él solo... Pero no pudo ser.
\newline

Y hoy llegas tú a mi vida, con tu sonrisa clara,\\
con tu sonrisa clara, que es un amanecer;\\
y ante el sueño más dulce que nunca antes soñara,\\
quiero vivir mi sueño... Pero no puede ser.
\newpage

Y he de decirte adiós para siempre, querida,\\
sabiendo que te alejas para nunca volver,\\
Quisiera retenerte para toda la vida...\\
¡Pero no puede ser! ¡Pero no puede ser!
\newline

José Ángel Buesa.
\end{verse}

\newpage
\begin{verse}
\begin{center}
\section{La dama de la rosa.}
\end{center}
Los que vieron la dama luciendo aquella rosa\\
que era como el fragante coágulo de una llama,\\
no supieron decirme cuál era más hermosa:\\
si la rosa o la dama.
\newline

Los que vieron la dama llevar la flor aquella,\\
como un broche de fuego sobre su piel sedosa,\\
no supieron decirme cual era la más bella:\\
si la dama o la rosa.
\newline

Cuando pasó la dama, fue un perfume su huella.\\
Nadie supo decirme si fue la flor, o ella,\\
la que dejó la noche perfumada.
\newline

Y yo, yo, que la tuve desnuda sobre el lecho,\\
yo, que corté la rosa para adornar su pecho,\\
tampoco dije nada.
\newline

José Ángel Buesa.
\end{verse}

\newpage
\begin{verse}
\begin{center}
\section{Canción del amor que se queda.}
\end{center}

\begin{center}
    I
\end{center}

Tu amor arde en la sombra como una llama lenta,\\
como la luz de un faro, que oscila en la tormenta.
\newline

Pérdida como el aire de la tarde en el trigo,\\
todo lo que me dejas también se va contigo.
\newline

Pérdida como el agua que salta de la fuente,\\
porque siempre es la misma y siempre es diferente;
\newline

y quizás tú te vas sin saber que te has ido,\\
como un golpe de viento, con un rumbo de olvido.
\newline

\begin{center}
    II
\end{center}

Yo he visto como el árbol recobra lo que pierde,\\
pues por cada hoja seca le brota una hoja verde;
\newline

pero también el árbol verdemente feliz\\
se seca hasta la copa si muere la raíz.
\newpage

\begin{center}
    III
\end{center}

Tu amor se va en la sombra como el agua de un río,\\
pero si el agua es tuya quizás el cauce es mío.
\newline

Tu amor es una alegre fugacidad de espuma\\
que se nutre del viento y en el viento se esfuma.
\newline

Pero es como una rama que florece, querida,\\
ver crecer en tus ojos una desconocida:
\newline

Ésa, recién llegada de tu ensueño o tu hastío,\\
nace en tu corazón, pero viene hacia el mío;
\newline

y si tú, como el agua que se va de una fuente,\\
siendo siempre la misma, puedes ser diferente,
\newline

yo, embriagado en tu vino con distinta embriaguez,\\
pensaré que eres otra, ¡para amarte otra vez!
\newline

José Ángel Buesa.
\end{verse}

\newpage
\begin{verse}
\begin{center}
\section{Poema para el crepúsculo.}
\end{center}
\begin{center}
I
\end{center}

Hora de soledad y de melancolía,\\
en que casi es de noche y casi no es de día.
\newline

Hora para que vuelva todo lo que se fue,\\
hora para estar triste, sin preguntar por qué.
\newline

Todo empieza a morir cuando nace el olvido.\\
Y es tan dulce buscar lo que no se ha perdido.
\newline

Y es tan agria esta angustia terriblemente cierta\\
de un gran amor dormido que de pronto despierta.

\begin{center}
II
\end{center}

Viendo pasar las nubes se comprende mejor\\
que así como ellas cambian, va cambiando el amor,
\newline

y aunque decimos: «Todo se olvida, todo pasa...»,\\
en las cenizas, a veces nos sorprende una brasa.
\newline

Porque es triste creer que se secó una fuente,\\
y que otro beba el agua que brota nuevamente;
\newpage

o una estrella apagada que vuelve a ser estrella,\\
y ver que hay otros ojos que están fijos en ella.
\newline

Decimos: «Todo pasa, porque todo se olvida»,\\
y el recuerdo entristece lo mejor de la vida.
\newline

\begin{center}
III
\end{center}

Apenas ha durado para amarte y perderte\\
este amor que debía durar hasta la muerte.
\newline

Fugaz como el contorno de una nube remota,\\
tu amor nace en la espiga muriendo en la gaviota.
\newline

Tu amor, cuando era mío, no me pertenecía.\\
Hoy, aunque vas con otro, quizás eres más mía.
\newline

Tu amor es como el viento que cruza de repente:\\
Ni se ve, ni se toca, pero existe y se siente.
\newline

Tu amor es como un árbol que renunció a su altura,\\
pero cuyas raíces abarcan la llanura.
\newline

Tu amor es como un viaje por el sueño de un loco,\\
porque nunca comienza ni termina tampoco. 
\newpage

Tu amor me negó siempre lo poco que pedí,\\
y hoy me da esta alegría de estar triste por ti.
\newline

Y, aunque creí olvidarte, pienso en ti todavía,\\
cuando, aún sin ser de noche, dejó de ser de día.
\newline

José Ángel Buesa.
\end{verse}


\newpage
\begin{verse}
\begin{center}
\section{Poema del río.}
\end{center}
Únicamente el río conoce tu secreto,\\
ese secreto tuyo que es el secreto mío.\\
El río es un hombre de corazón inquieto\\
pero el amor se aleja como el agua del río.
\newline

Únicamente el río nos vio por la vereda,\\
y el rumor de sus aguas era como un reproche.\\
Tu piel era más blanca bajo la magra seda,\\
como el deslumbramiento de la nieve en la noche.
\newline

No importa que huya el agua como un amor de un día;\\
mi amor, igual que el río, se quedará aunque huya.\\
Únicamente el río supo que fuiste mía,\\
para que mi alma fuera profundamente tuya.
\newline

El río es como un viaje para el sueño del hombre,\\
el hombre, es como el río, un gran dolor en viaje.\\
Únicamente el río te oyó decir mí nombre\\
cuando las hojas secas decoraron tu traje.
\newpage

Sí. El río es como un hombre de corazón inquieto\\
que va encendiendo hogueras y se muere de frío.\\
Únicamente el río conoce tu secreto.\\
Únicamente el río.
\newline

José Ángel Buesa.
\end{verse}

\newpage
\begin{verse}
\begin{center}
\section{Segundo poema en la alameda.}
\end{center}
No sé por que he venido de nuevo a la alameda.\\
Tú no la conocías. Yo, casi no la conozco.\\
Y, sin embargo, un día me embriagué de ternura\\
bajo estas frondas quietas, entre estos viejos troncos.
\newline

Hoy, que sé que jamás he de volver con ella,\\
con la que todavía me entristece los ojos;\\
hoy, que ya para siempre nos separa la vida,\\
vengo contigo, acaso para no venir solo...
\newline

Aquí todo ha cambiado, como yo, como ella...\\
Los pájaros volaron con el viento de otoño,\\
y entre las hojas secas que caen en la tarde,\\
el eco de sus pasos va surgiendo del  polvo...
\newline

Y tú vienes conmigo... Tú, que quizás me quieres,\\
y que quizás me olvides pronto;\\
con tu chaqueta gris y tus ojos azules\\
te apoyas en mi brazo, bajo el crepúsculo de oro.
\newline

Seis veces estos árboles se han quedado sin hojas\\
desde la última vez... Seis veces: es bien poco.\\
Y, aunque quizás no haya cambiado nada,\\
hoy vuelvo, y me parece que es diferente todo.
\newpage

Aquí, junto a esta verja, le di el último beso.\\
Yo entonces era soñador y loco,\\
y todavía entonces me sonreía sin motivo,\\
y mi alma era una playa frente a un océano sonoro.
\newline

Ya apenas la recuerdo, pero nunca la olvido.\\
Nos separó la vida... así, sin saber cómo.\\
Y hoy, tú que no eres ella,\\
te apoyas en mi brazo, que es casi el brazo de otro...
\newline

José Ángel Buesa.
\end{verse}

\newpage
\begin{verse}
\begin{center}
\section{Canción de la hoguera.}
\end{center}
Diré que junto a un árbol resplandece una hoguera,\\
y que estará encendida mañana igual que ayer...\\
En invierno y otoño, verano y primavera,\\
arde esa hoguera loca sin que deje de arder.
\newline

Le dio sus hojas secas el árbol corpulento;\\
después, las hojas verdes, y los gajos quizás...\\
Y aunque es mayor la llama cuando la sopla el viento\\
no importa si arde pronto, porque ilumina más.
\newline

Y no importa si el árbol no tiene flor ni fruto,\\
porque muere en el sueño de una muerte feliz:\\
y cuando falten ramas para el fugaz tributo,\\
convertirá en cenizas, su tronco y su raíz...
\newline

Mas, si alguien no comprende la verdad escondida\\
en la hoguera implacable y en el árbol sin flor,\\
yo le diré que el árbol que se quema es mi vida,\\
y que la hoguera es el amor.
\newline

José Ángel Buesa.
\end{verse}

\newpage
\begin{verse}
\begin{center}
\section{Poema final.}
\end{center}
Yo cantaré algún día la angustia verdadera,\\
y, así lo que otros callan lo iré diciendo yo,\\
pues la mujer que amamos sin que ella lo supiera,\\
sin saberlo nosotros, acaso nos amó...
\newline

Aunque el tiempo nos lleva por un camino triste,\\
mientras tu cuerpo avanza, tu alma puede volver,\\
porque, en tu amor de ahora, tu amor de ayer subsiste,\\
y en la mujer que hoy amas sonríe otra mujer.
\newline

Y es que el amor más grande nos parece pequeño\\
mientras haya otra boca que podamos besar,\\
y el corazón no sabe la medida del sueño\\
como nadie ha sabido la medida del mar.
\newline

Porque el alma inconforme pide más a la vida,\\
que en cada don que otorga nos arrebata un don,\\
y así nos mata un sueño con cada despedida\\
y nos cuenta una muerte cada resurrección.
\newline

Pero el amor sonríe como un niño dormido,\\
y el mañana es la sombra de la luz del ayer;\\
y así se va la vida, sin saber que se ha ido,\\
como se van las nubes en el atardecer...
\newpage

Y ahora, yo, que he hecho mía toda esa angustia ajena,\\
que canté sonriendo lo triste del azar,\\
comprendo que he cantado también mi propia pena,\\
y que he dicho las cosas que quería callar.
\newline

José Ángel Buesa.
\end{verse}

\newpage
\begin{verse}
\begin{center}
\section{La puerta.}
\end{center}
\begin{center}
I
\end{center}

Recuerdo bien que te cerré la puerta.\\
Sé que llamaste, y sé que no te abrí...\\
Y ahora miro la puerta, y está abierta,\\
y te siento de pronto junto a mí.
\newline

Entraste, y no sé cómo todavía;\\
pero sé que este amor tiene que ser\\
como la claridad del mediodía\\
en la penumbra del anochecer.
\newline

Y es tan inesperado este cariño\\
que lo rechazo y lo retengo al par,\\
como una madre que reprende a un niño,\\
pero qué llora viéndolo llorar...
\newline

\begin{center}
II
\end{center}

Has entrado en mi amor tan silenciosa,\\
que no sentí ni el roce de tu pie;\\
y eres como el milagro de la rosa,\\
que se hace rosa sin saber por qué...
\newpage

Y me penetra tu emoción sencilla,\\
más allá de mi bien y de mi mal,\\
como la gota de agua por la arcilla,\\
como la luz del sol por un cristal.
\newline

Y, cada vez más hondo, en lo más puro,\\
tu amor se hace el camino de mi amor,\\
como la hiedra que se ciñe al muro,\\
pero que lo reviste de verdor.

\begin{center}
III
\end{center}

Yo te cerré la puerta, y tú la abriste,\\
y te acercaste a mí con timidez,\\
con tu sonrisa de muchacha triste\\
que va a una fiesta por primera vez.
\newline

Y ahora sé que el amor entró contigo,\\
mujer que, hecha de amor y para amar,\\
tienes la doble cualidad del trigo:\\
pan en la mesa y carne en el altar.
\newline

Y ahora me da temor la puerta abierta,\\
aunque por ella entró el amanecer...\\
Pero esta vez voy a cerrar la puerta\\
para que no te puedas ir, mujer.
\newline

José Ángel Buesa.
\end{verse}

\newpage
\begin{verse}
\begin{center}
\section{Envío.}
\end{center}
La vida pasa; la vida rueda...\\
Quizás se aparten tu alma y la mía,\\
pero el recuerdo nace y se queda...\\
Y aunque el deseo no retroceda\\
y nuestra llama se apague un día,\\
mientras yo pueda soñar, y pueda\\
regar mis sueños en la vereda\\
de la armonía,\\
tendré la dulce melancolía\\
de aquellas frases entre la umbría\\
y aquellos besos en la alameda.

José Ángel Buesa.
\end{verse}


\newpage
\begin{verse}
\begin{center}
\section{Poema del olvido.}
\end{center}
Viendo pasar las nubes fue pasando la vida,\\
y tú, como una nube, pasaste por mi hastío.\\
Y se unieron entonces tu corazón y el mío,\\
como se van uniendo los bordes de una herida.
\newline

Los últimos ensueños y las primeras canas\\
entristecen de sombra todas las cosas bellas;\\
y hoy tu vida y mi vida son como estrellas,\\
pues pueden verse juntas, estando tan lejanas...
\newline

Yo bien sé que el olvido, como un agua maldita,\\
nos da una sed más honda que la sed que nos quita,\\
pero estoy tan seguro de poder olvidar...
\newline

Y miraré las nubes sin pensar que te quiero,\\
con el hábito sordo de un viejo marinero\\
que aún siente, en tierra firme, la ondulación del mar.
\newline

José Ángel Buesa.
\end{verse}

\newpage
\begin{verse}
\begin{center}
\section{Elegía de estío.}
\end{center}
\begin{center}
I
\end{center}

Aquí, junto a este río, tu ausencia es más ausente.\\
Aquí, bajo este árbol, mi silencio es más mío.
\newline

Algo de mi alegría se fue en esta corriente,\\
o dejó en estas ramas un retoño tardío.
\newline

Ahora pasan el tiempo y el agua transparente\\
por este cauce turbio y amargo de mi hastío.
\newline

\begin{center}
II
\end{center}

Pareciéndote a un río, que al irse no está ausente,\\
te pareces a un árbol con un nido vacío.
\newline

Todo está como entonces, y todo es diferente,\\
porque faltan tus ojos en la tarde de estío.
\newline

Y el amor que llegaba se nos fue dulcemente,\\
con la sombra del árbol, en el agua del río.
\newline

José Ángel Buesa.
\end{verse}

\newpage
\begin{verse}
\begin{center}
\section{Te irás, tal vez.}
\end{center}
Te irás, tal vez; te irás, como una barca\\
buscando el mar huyendo de la tierra,\\
pero estarás en mi, como la marca\\
de un doblez en un libro que se cierra.
\newline

Te irás, tal vez; y como tantas cosas\\
que están presentes aunque se hayan ido,\\
serás en mí como un rosal sin rosas\\
pero secretamente florecido.
\newline

Te irás, tal vez; te irás calladamente,\\
mas si el humo se va, queda la brasa,\\
y te parecerás a la corriente\\
que, pasando y pasando, nunca pasa….
\newline

Y así te irás sin irte, como un largo\\
rumor de agua cayendo noche y día,\\
pues deja de llover, y sin embargo,\\
nos parece que llueve todavía...
\newline

José Ángel Buesa.
\end{verse}

\newpage
\begin{verse}
\begin{center}
\section{Ésta vieja canción.}
\end{center}
Esta vieja canción que oí contigo,\\
y que contigo di por olvidada,\\
surge del fondo de la madrugada\\
como la voz doliente de un amigo.
\newline

(Yo sé que la mujer que va contigo\\
no puede adivinar en mi mirada\\
que esa canción que no le dice nada,\\
le está diciendo lo que yo no digo).
\newline

Y, al escuchar de pronto esa tonada,\\
comprendo la amargura de un mendigo\\
ante una puerta que le fue cerrada.
\newline

Pero intento reír, y lo consigo...\\
como si no me recordara nada\\
esta vieja canción que oí contigo.
\newline

José Ángel Buesa.
\end{verse}

\newpage
\begin{verse}
\begin{center}
\section{Soneto adolescente.}
\end{center}
Qué dulce, si lloviera de repente...\\
No sé por qué, porque tú estás lejana,\\
pero en la soledad de esta mañana\\
hay algo de tu amor que no está ausente.
\newline

Y yo sonrío, extraño adolescente\\
de ojos cansados y cabeza cana,\\
yo, que aún puedo asomarme a la ventana\\
y ver la luna que no ve la gente...
\newline

Ah, sí, qué dulcemente llovería\\
con ese sol, para olvidar un poco\\
mi prematura gran pasión tardía...
\newline

Y yo cierro los párpados huraños\\
pensando en ti, yo, extravagante y loco\\
adolescente de cuarenta años.
\newline

José Ángel Buesa.
\end{verse}

\newpage
\begin{verse}
\begin{center}
\section{Elegía nocturna.}
\end{center}
Quién nos hubiera dicho que todo acabaría,\\
como acaba en la sombra la claridad del día.
\newline

Fuiste como la lluvia cayendo sobre un río,\\
para que fuera tuyo todo lo que era mío.
\newline

Fuiste como una lámpara que se encendió en mi vida;\\
yo la soplé de pronto, pero siguió encendida.
\newline

Fuiste un río ilusorio, cantando en un desierto;\\
y floreció la arena como si fuera cierto.
\newline

Mi amor fue una gaviota que construyó su nido,\\
en lo alto de un mástil; ahora el buque se ha ido.
\newline

Ahora me envuelve un hosco silencio de campana,\\
donde sólo resuena tu campana lejana.
\newline

Como un surco amargo que se negara al trigo,\\
ahora mi alma no sueña, por no soñar contigo.
\newline

José Ángel Buesa.
\end{verse}

\newpage
\begin{verse}
\begin{center}
\section{Pequeña canción II.}
\end{center}
Aún alegran tu calle los viejos mediodías\\
y la sombra del álamo refresca tu portal,\\
todo está como entonces, cuando tú me querías,\\
pero ya no me quieres, y todo sigue igual.
\newline

Sin embargo, no importa, yo sé que me quisiste\\
más allá de aquel beso, de aquel que no te di,\\
y sé que alguna noche te irás quedando triste\\
al ponerte un vestido que me gustaba a mí.
\newline

José Ángel Buesa.
\end{verse}

\newpage
\begin{verse}
\begin{center}
\section{Poema para olvidarte.}
\end{center}
Amar —nadie lo ignora— viene a ser como un juego:\\
el juego de dos almas y el juego de dos vidas.\\
Y hay quien gana y quién pierde. Tal vez lo sabrás luego,\\
si yo logro olvidarte pero tú no me olvidas.
\newline

Yo sé por qué lo digo. La vida tiene un modo\\
sutil de detenerse mientras sigue adelante,\\
y una mujer bonita puede olvidarlo todo\\
menos su última cita con su primer amante.
\newline

Por eso, allá... tan lejos... en tus tardes de hastío,\\
puede ser que comprendas que el hombre a quien quisiste\\
llenó de mariposas tu corazón vacío\\
y de fechas alegres tu calendario triste.
\newline

Y como tu pasado no pasó todavía\\
tendrás que recordarme viendo en tu tocador\\
aquellos espejuelos oscuros con que un día\\
disimulaste un poco tus tijeras de amor.
\newpage

Y yo sé que otro día, de rezos y conjuros,\\
te dirán que me he muerto  —yo sé que será así —\\
y te pondrás los mismos espejuelos oscuros\\
para que nadie sepa que lloraste por mí.
\newline

José Ángel Buesa.
\end{verse}

\newpage
\begin{verse}
\begin{center}
\section{La estrella.}
\end{center}
Yo sigo enamorado de la estrella\\
que ilumina mi melancolía\\
dándole miel a la ternura aquella\\
que acaso era vulgar, pero era mía.
\newline

Mi corazón ha envejecido un poco,\\
pero, a pesar de su envejecimiento,\\
me duele todavía si lo toco\\
y todavía se me va en el viento.
\newline

Y tercamente, qué sé yo hasta cuándo,\\
mi viejo corazón sigue esperando\\
la última rosa del jardín marchito;
\newline

y ya después no importa que se vaya,\\
como la última arena de una playa,\\
con el último verso que haya escrito.
\newline

José Ángel Buesa.
\end{verse}

\newpage
\begin{verse}
\begin{center}
\section{Canción de la rosa.}
\end{center}
Hay que cortar la rosa, pues de cualquier manera\\
se secará en la rama su adorable ornamento;\\
y, al renacer cien veces con cada primavera,\\
es cien veces más triste que la deshoje el viento.
\newline

Hay que cortar la rosa, pues siempre se termina\\
fugazmente su encanto para aquel que lo ama,\\
y al final sobrevive solamente la espina,\\
que es también lo primero que le nace a la rama.
\newline

Por eso, en esta angustia de andar hacia el olvido,\\
lúgubres caminantes de la noche luctuosa,\\
para no lamentarnos del tiempo que se ha ido\\
hay que cerrar los ojos y hay que cortar la rosa...
\newline

José Ángel Buesa.
\end{verse}


\newpage
\begin{verse}
\begin{center}
\section{Canción al olvido.}
\end{center}
Aquel amor que se nos fuera\\
no lo debemos recordar:\\
Árbol que muere en primavera\\
ya nunca vuelve a  retoñar.
\newline

Perla que en el humo se disuelve,\\
peregrina de la emoción,\\
la ilusión que se va, no vuelve\\
jamás a nuestro corazón.
\newline

Vanamente, pretenderemos\\
dar a una rosa mustia color.\\
Así tampoco logramos\\
dar nueva vida a un muerto amor.
\newline

Aquel amor que se nos fuera\\
no lo debemos recordar:\\
Árbol que muere en primavera\\
ya nunca vuelve a  retoñar.
\newline

Cuando el amor se siente extraño\\
en el pecho, ya no es amor,\\
y retenerlo es un engaño\\
que tortura al engañador...
\newpage

Déjalo ir... deja vacío\\
ese hueco en  tu corazón,\\
en las cenizas de tu hastío\\
pon la brasa de otra ilusión...
\newline

Aquel amor que se nos fuera\\
no lo debemos recordar:\\
Árbol que muere en primavera\\
ya nunca vuelve a retoñar...
\newline

Muerto está el amor al que envuelve\\
en llamas la imaginación:\\
La ilusión que se va, no vuelve\\
jamás a nuestro corazón.
\newline

Es ley amarga de la vida\\
de todo sueño despertar:\\
Sobre las huellas de una huida\\
es inútil querer soñar...
\newline

Así, triste, pero sumisa,\\
aceptando el dolor, mujer,\\
di adiós con tu mejor sonrisa\\
a lo que nunca ha de volver...
\newpage

Enigma que si se resuelve\\
nos desencanta, es la pasión:\\
La ilusión que se va, no vuelve\\
jamás a nuestro corazón...
\newline

Juntemos, pues, las manos frías,\\
y digamos una oración\\
por las pasadas alegrías\\
y por la actual desilusión.
\newline

Y con humilde voz, pidamos\\
pronto consuelo a este dolor,\\
por lo mucho que nos amamos\\
en lo breve de nuestro amor...
\newline

Como la mar, no vuelve\\
al río su agua, la ilusión,\\
una vez que se va, no vuelve\\
jamás a nuestro corazón.
\newpage

Aquel amor que se nos fuera\\
no lo debemos recordar:\\
¡Árbol que muere en primavera\\
ya nunca vuelve a retoñar!..
\newline

Hay que vivir, hay que olvidar...
\newline

José Ángel Buesa.
\end{verse}

\newpage
\begin{verse}
\begin{center}
\section{Canción contigo.}
\end{center}
Aquí estás en la sombra,\\
con tu mano en la mía,\\
respirando en un tiempo\\
sin antes ni después.
\newline

Ya ves que,\\
aunque te fuiste,\\
no te vas todavía,\\
y estas aquí, conmigo\\
no importa donde estés.
\newline

Desnuda en esta sombra\\
te palpará mi mano,\\
lenta mano de ciego\\
que acaricia una flor,\\
y sabré de repente\\
donde empieza el verano,\\
yo, que solo he sabido\\
donde acaba el amor.
\newline

Aquí estás en la sombra,\\
conmigo todavía,\\
compartiendo este lecho\\
cálidamente aquí,\\
Detenida en la noche,\\
y donde nunca es de día,\\
detenida en la noche\\
y amaneciendo en mí.
\newline

Y ahora soy como el surco\\
donde madura el trigo,\\
como la flor que nace\\
donde pisan tus pies,\\
porque, aunque nunca vuelvas,\\
siempre estarás conmigo,\\
conmigo en esta sombra\\
sin antes ni después.
\newline

José Ángel Buesa.
\end{verse}

\newpage
\begin{verse}
\begin{center}
\section{Canción agradecida.}
\end{center}
Gracias, amor, si hiciste que lloviera\\
en el último instante de este día,\\
pues, por ser una lluvia triste y fría,\\
hubo un rayo de sol sobre una hoguera.
\newline

Gracias, amor, si tu designio era\\
que lloviera del modo que llovía\\
para ofrecerme en una flor tardía\\
todo el perfume de la primavera.
\newline

Gracias, amor, si no la merecía,\\
gracias, amor, aunque la mereciera;\\
gracias también por la melancolía.
\newline

Que llueve dentro cuando escampa afuera,\\
y haz que vuelva a llover de esa manera\\
como llueve en mi alma todavía.
\newline

José Ángel Buesa.
\end{verse}

\newpage
\begin{verse}
\begin{center}
\section{Soneto final.}
\end{center}
Y cerraré los ojos para siempre, algún día\\
y habrá noches de estrellas que ya nunca he de ver\\
y cantará otra boca lo que cantó la mía,\\
cuando pasan las nubes en el atardecer.
\newline

Y habrá polvo en los bordes de la copa vacía\\
donde exalté mi ensueño y aturdí mi placer.\\
Y en las tardes de otoño lloverá todavía,\\
para que otro hombre triste recuerde a otra mujer.
\newline

Todo será lo mismo, y a la vez diferente,\\
habrá rosas y besos naciendo dulcemente\\
y un niño sin infancia caminando hacia el mar...
\newline

Y yo seré la sombra de un viajero tardío\\
que quiso ser el cauce donde pasara un río,\\
y fue solo una nube que no volvió a pasar.
\newline

José Ángel Buesa.
\end{verse}

\newpage
\begin{verse}
\begin{center}
\section{Veinte años amiga.}
\end{center}
Veinte años, amiga. Y hoy al verte de lejos,\\
evoqué a la muchacha gentil de mi canción.\\
Y aprendí, en un suspiro, que vamos siendo viejos,\\
aunque nunca envejezca del todo el corazón.
\newline

Veinte años, amiga. Y al decir ``veinte años",\\
mi corazón añade: ``separado de ti..."\\
Y pensar que hoy nos vemos igual que dos extraños;\\
y saber que las rosas se marchitan así...
\newline

Veinte años, amiga. Cómo duele el olvido.\\
Pero las cosas pasan y queda la ilusión;\\
y, aunque con tu belleza, tu juventud se ha ido,\\
tú sigues siendo joven y bella en mi canción...
\newline

José Ángel Buesa.
\end{verse}

\newpage
\begin{verse}
\begin{center}
\section{La rama rota.}
\end{center}
Vengo de tu jardín de altos aromas,\\
con esta flor que embriaga como un vino.\\
Quizás por eso fue que en el camino\\
me siguió una bandada de palomas.
\newline

Y ahora, en mi huerto, en esta entristecida\\
paz del que nada odia y nada ama,\\
me tropiezan los pies con una rama\\
seca y rota, lo mismo que mi vida.
\newline

Y, como quien regresa del olvido\\
y se hermana al dolor de otra derrota,\\
pongo la flor sobre la rama rota\\
para hacerle creer que ha florecido.
\newline

José Ángel Buesa.
\end{verse}


\newpage
\begin{verse}
\begin{center}
\section{Bendita seas.}
\end{center}
Bendita seas...\\
Fuiste algo blanco, muy blanco y puro,\\
en la agonía del hierro oscuro\\
donde se abrían las negras rosas de mis ideas...
\newline

Porque al amarme desvaneciste\\
mis negaciones hondas y ateas;\\
porque eres buena, porque eres triste,\\
bendita seas.
\newline

Porque endulzaste mis desalientos,\\
porque encantaste mis desencantos,\\
porque elevaste mis pensamientos;\\
porque al mirarme tus ojos santos\\
se iluminaron mis sufrimientos\\
y mis quebrantos;
\newline

porque curaste, caritativa,\\
todas las llagas de mis peleas;\\
por delicada, por comprensiva,\\
bendita seas...
\newpage

porque tú fuiste como un remanso\\
para el estruendo de mis mareas;\\
porque me diste paz y descanso,\\
¡bendita seas!
\newline

Hoy voy de nuevo por el camino\\
do en polvo escriben mi vida inquieta\\
mis pies llagados de peregrino,\\
oyendo a un ave de dulce trino\\
que rima versos como un poeta,\\
y viendo siempre la gris silueta\\
de mi destino...
\newline

pero, en la hora de la parida,\\
cuando sus fauces abre lo arcano,\\
y, como un ala, tiembla en la mano\\
la despedida;
\newpage

cuando mi viaje sin rumbo emprendo,\\
ensombrecido por el estruendo\\
de mis mareas;\\
cuando de nuevo mi andanza sigo,\\
porque me amaste, porque me diste\\
las dulcedumbres de tu alma triste,\\
yo te bendigo...\\
¡Bendita seas!
\newline

José Ángel Buesa.
\end{verse}

\end{document}